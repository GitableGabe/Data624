% Options for packages loaded elsewhere
\PassOptionsToPackage{unicode}{hyperref}
\PassOptionsToPackage{hyphens}{url}
\PassOptionsToPackage{dvipsnames,svgnames,x11names}{xcolor}
%
\documentclass[
]{article}
\usepackage{amsmath,amssymb}
\usepackage{iftex}
\ifPDFTeX
  \usepackage[T1]{fontenc}
  \usepackage[utf8]{inputenc}
  \usepackage{textcomp} % provide euro and other symbols
\else % if luatex or xetex
  \usepackage{unicode-math} % this also loads fontspec
  \defaultfontfeatures{Scale=MatchLowercase}
  \defaultfontfeatures[\rmfamily]{Ligatures=TeX,Scale=1}
\fi
\usepackage{lmodern}
\ifPDFTeX\else
  % xetex/luatex font selection
\fi
% Use upquote if available, for straight quotes in verbatim environments
\IfFileExists{upquote.sty}{\usepackage{upquote}}{}
\IfFileExists{microtype.sty}{% use microtype if available
  \usepackage[]{microtype}
  \UseMicrotypeSet[protrusion]{basicmath} % disable protrusion for tt fonts
}{}
\makeatletter
\@ifundefined{KOMAClassName}{% if non-KOMA class
  \IfFileExists{parskip.sty}{%
    \usepackage{parskip}
  }{% else
    \setlength{\parindent}{0pt}
    \setlength{\parskip}{6pt plus 2pt minus 1pt}}
}{% if KOMA class
  \KOMAoptions{parskip=half}}
\makeatother
\usepackage{xcolor}
\usepackage[margin=1in]{geometry}
\usepackage{color}
\usepackage{fancyvrb}
\newcommand{\VerbBar}{|}
\newcommand{\VERB}{\Verb[commandchars=\\\{\}]}
\DefineVerbatimEnvironment{Highlighting}{Verbatim}{commandchars=\\\{\}}
% Add ',fontsize=\small' for more characters per line
\usepackage{framed}
\definecolor{shadecolor}{RGB}{248,248,248}
\newenvironment{Shaded}{\begin{snugshade}}{\end{snugshade}}
\newcommand{\AlertTok}[1]{\textcolor[rgb]{0.94,0.16,0.16}{#1}}
\newcommand{\AnnotationTok}[1]{\textcolor[rgb]{0.56,0.35,0.01}{\textbf{\textit{#1}}}}
\newcommand{\AttributeTok}[1]{\textcolor[rgb]{0.13,0.29,0.53}{#1}}
\newcommand{\BaseNTok}[1]{\textcolor[rgb]{0.00,0.00,0.81}{#1}}
\newcommand{\BuiltInTok}[1]{#1}
\newcommand{\CharTok}[1]{\textcolor[rgb]{0.31,0.60,0.02}{#1}}
\newcommand{\CommentTok}[1]{\textcolor[rgb]{0.56,0.35,0.01}{\textit{#1}}}
\newcommand{\CommentVarTok}[1]{\textcolor[rgb]{0.56,0.35,0.01}{\textbf{\textit{#1}}}}
\newcommand{\ConstantTok}[1]{\textcolor[rgb]{0.56,0.35,0.01}{#1}}
\newcommand{\ControlFlowTok}[1]{\textcolor[rgb]{0.13,0.29,0.53}{\textbf{#1}}}
\newcommand{\DataTypeTok}[1]{\textcolor[rgb]{0.13,0.29,0.53}{#1}}
\newcommand{\DecValTok}[1]{\textcolor[rgb]{0.00,0.00,0.81}{#1}}
\newcommand{\DocumentationTok}[1]{\textcolor[rgb]{0.56,0.35,0.01}{\textbf{\textit{#1}}}}
\newcommand{\ErrorTok}[1]{\textcolor[rgb]{0.64,0.00,0.00}{\textbf{#1}}}
\newcommand{\ExtensionTok}[1]{#1}
\newcommand{\FloatTok}[1]{\textcolor[rgb]{0.00,0.00,0.81}{#1}}
\newcommand{\FunctionTok}[1]{\textcolor[rgb]{0.13,0.29,0.53}{\textbf{#1}}}
\newcommand{\ImportTok}[1]{#1}
\newcommand{\InformationTok}[1]{\textcolor[rgb]{0.56,0.35,0.01}{\textbf{\textit{#1}}}}
\newcommand{\KeywordTok}[1]{\textcolor[rgb]{0.13,0.29,0.53}{\textbf{#1}}}
\newcommand{\NormalTok}[1]{#1}
\newcommand{\OperatorTok}[1]{\textcolor[rgb]{0.81,0.36,0.00}{\textbf{#1}}}
\newcommand{\OtherTok}[1]{\textcolor[rgb]{0.56,0.35,0.01}{#1}}
\newcommand{\PreprocessorTok}[1]{\textcolor[rgb]{0.56,0.35,0.01}{\textit{#1}}}
\newcommand{\RegionMarkerTok}[1]{#1}
\newcommand{\SpecialCharTok}[1]{\textcolor[rgb]{0.81,0.36,0.00}{\textbf{#1}}}
\newcommand{\SpecialStringTok}[1]{\textcolor[rgb]{0.31,0.60,0.02}{#1}}
\newcommand{\StringTok}[1]{\textcolor[rgb]{0.31,0.60,0.02}{#1}}
\newcommand{\VariableTok}[1]{\textcolor[rgb]{0.00,0.00,0.00}{#1}}
\newcommand{\VerbatimStringTok}[1]{\textcolor[rgb]{0.31,0.60,0.02}{#1}}
\newcommand{\WarningTok}[1]{\textcolor[rgb]{0.56,0.35,0.01}{\textbf{\textit{#1}}}}
\usepackage{graphicx}
\makeatletter
\def\maxwidth{\ifdim\Gin@nat@width>\linewidth\linewidth\else\Gin@nat@width\fi}
\def\maxheight{\ifdim\Gin@nat@height>\textheight\textheight\else\Gin@nat@height\fi}
\makeatother
% Scale images if necessary, so that they will not overflow the page
% margins by default, and it is still possible to overwrite the defaults
% using explicit options in \includegraphics[width, height, ...]{}
\setkeys{Gin}{width=\maxwidth,height=\maxheight,keepaspectratio}
% Set default figure placement to htbp
\makeatletter
\def\fps@figure{htbp}
\makeatother
\setlength{\emergencystretch}{3em} % prevent overfull lines
\providecommand{\tightlist}{%
  \setlength{\itemsep}{0pt}\setlength{\parskip}{0pt}}
\setcounter{secnumdepth}{-\maxdimen} % remove section numbering
\ifLuaTeX
  \usepackage{selnolig}  % disable illegal ligatures
\fi
\IfFileExists{bookmark.sty}{\usepackage{bookmark}}{\usepackage{hyperref}}
\IfFileExists{xurl.sty}{\usepackage{xurl}}{} % add URL line breaks if available
\urlstyle{same}
\hypersetup{
  pdftitle={DATA 624: PREDICTIVE ANALYTICS HW3},
  pdfauthor={Gabriel Campos},
  colorlinks=true,
  linkcolor={Maroon},
  filecolor={Maroon},
  citecolor={Blue},
  urlcolor={blue},
  pdfcreator={LaTeX via pandoc}}

\title{DATA 624: PREDICTIVE ANALYTICS HW3}
\author{Gabriel Campos}
\date{Last edited February 18, 2024}

\begin{document}
\maketitle

\hypertarget{instructions}{%
\section{Instructions}\label{instructions}}

Do exercises 5.1, 5.2, 5.3, 5.4 and 5.7 in the Hyndman book. Please
submit your \href{https://rpubs.com/gcampos100/DATA_624_HW3}{Rpubs link}
as well as your .pdf file showing your run code.

\begin{Shaded}
\begin{Highlighting}[]
\FunctionTok{library}\NormalTok{(dplyr)}
\FunctionTok{library}\NormalTok{(stringr)}
\FunctionTok{library}\NormalTok{(fpp3)}
\FunctionTok{library}\NormalTok{(cowplot)}
\end{Highlighting}
\end{Shaded}

\hypertarget{section}{%
\section{5.1}\label{section}}

Produce forecasts for the following series using whichever of
\texttt{NAIVE(y)}, \texttt{SNAIVE(y)} or
\texttt{RW(y\ \textasciitilde{}\ drift())} is more appropriate in each
case:

\hypertarget{i}{%
\subsection{i}\label{i}}

Australian Population (\texttt{global\_economy})

\begin{Shaded}
\begin{Highlighting}[]
\NormalTok{df\_aus }\OtherTok{\textless{}{-}}\NormalTok{ global\_economy }\SpecialCharTok{\%\textgreater{}\%}
            \FunctionTok{filter}\NormalTok{(Country }\SpecialCharTok{==} \StringTok{"Australia"}\NormalTok{)}

\FunctionTok{head}\NormalTok{(df\_aus)}
\end{Highlighting}
\end{Shaded}

\begin{verbatim}
## # A tsibble: 6 x 9 [1Y]
## # Key:       Country [1]
##   Country   Code   Year          GDP Growth   CPI Imports Exports Population
##   <fct>     <fct> <dbl>        <dbl>  <dbl> <dbl>   <dbl>   <dbl>      <dbl>
## 1 Australia AUS    1960 18573188487.  NA     7.96    14.1    13.0   10276477
## 2 Australia AUS    1961 19648336880.   2.49  8.14    15.0    12.4   10483000
## 3 Australia AUS    1962 19888005376.   1.30  8.12    12.6    13.9   10742000
## 4 Australia AUS    1963 21501847911.   6.21  8.17    13.8    13.0   10950000
## 5 Australia AUS    1964 23758539590.   6.98  8.40    13.8    14.9   11167000
## 6 Australia AUS    1965 25931235301.   5.98  8.69    15.3    13.2   11388000
\end{verbatim}

\begin{Shaded}
\begin{Highlighting}[]
\NormalTok{ aus\_plot1 }\OtherTok{\textless{}{-}}\NormalTok{ df\_aus}\SpecialCharTok{\%\textgreater{}\%}
  \FunctionTok{autoplot}\NormalTok{(Population)}\SpecialCharTok{+}
  \FunctionTok{labs}\NormalTok{(}\AttributeTok{title=} \StringTok{"Australian Population"}\NormalTok{)}\SpecialCharTok{+}
  \FunctionTok{annotate}\NormalTok{(}\StringTok{"text"}\NormalTok{, }\AttributeTok{x =} \ConstantTok{Inf}\NormalTok{, }\AttributeTok{y =} \SpecialCharTok{{-}}\ConstantTok{Inf}\NormalTok{, }\AttributeTok{hjust =} \DecValTok{1}\NormalTok{, }\AttributeTok{vjust =} \SpecialCharTok{{-}}\DecValTok{1}\NormalTok{,}
           \AttributeTok{label =} \StringTok{"There is an upward trend"}\NormalTok{)}

\NormalTok{aus\_fit }\OtherTok{\textless{}{-}}\NormalTok{ df\_aus }\SpecialCharTok{\%\textgreater{}\%}
            \CommentTok{\# no filter needed}
            \FunctionTok{model}\NormalTok{(}\FunctionTok{RW}\NormalTok{(Population }\SpecialCharTok{\textasciitilde{}} \FunctionTok{drift}\NormalTok{()))}

\NormalTok{aus\_fc }\OtherTok{\textless{}{-}}\NormalTok{ aus\_fit }\SpecialCharTok{\%\textgreater{}\%}
            \FunctionTok{forecast}\NormalTok{(}\AttributeTok{h =} \DecValTok{10}\NormalTok{)}

\NormalTok{aus\_plot2}\OtherTok{\textless{}{-}}\NormalTok{ aus\_fc }\SpecialCharTok{\%\textgreater{}\%} 
  \FunctionTok{autoplot}\NormalTok{(df\_aus)}



\FunctionTok{plot\_grid}\NormalTok{(aus\_plot1, aus\_plot2, }\AttributeTok{ncol =} \DecValTok{2}\NormalTok{)}
\end{Highlighting}
\end{Shaded}

\includegraphics{Gabriel_Campos_DATA624_HW3_files/figure-latex/unnamed-chunk-3-1.pdf}

Alternatively the plotting can be done with one function

\begin{Shaded}
\begin{Highlighting}[]
\NormalTok{df\_aus}\SpecialCharTok{\%\textgreater{}\%}
  \FunctionTok{model}\NormalTok{(}\FunctionTok{RW}\NormalTok{(Population }\SpecialCharTok{\textasciitilde{}} \FunctionTok{drift}\NormalTok{()))}\SpecialCharTok{\%\textgreater{}\%}
    \FunctionTok{forecast}\NormalTok{(}\AttributeTok{h =} \DecValTok{10}\NormalTok{)}\SpecialCharTok{\%\textgreater{}\%}
      \FunctionTok{autoplot}\NormalTok{(df\_aus)}
\end{Highlighting}
\end{Shaded}

\includegraphics{Gabriel_Campos_DATA624_HW3_files/figure-latex/unnamed-chunk-4-1.pdf}

In section 5.2 Drift method is explained to ``allow the forecasts to
increase or decrease over time''. Since the data did not show high
seasonality and was not economic or financial, I did not use
\texttt{Naive} method and \texttt{seasonal\ naive}. The example shown
also uses random walk forecast in conjunction with drift (refer below)

\begin{verbatim}
                    bricks |> model(RW(Bricks ~ drift()))
\end{verbatim}

\hypertarget{ii}{%
\subsection{ii}\label{ii}}

Bricks (\texttt{aus\_production})

\begin{Shaded}
\begin{Highlighting}[]
\FunctionTok{head}\NormalTok{(aus\_production)}
\end{Highlighting}
\end{Shaded}

\begin{verbatim}
## # A tsibble: 6 x 7 [1Q]
##   Quarter  Beer Tobacco Bricks Cement Electricity   Gas
##     <qtr> <dbl>   <dbl>  <dbl>  <dbl>       <dbl> <dbl>
## 1 1956 Q1   284    5225    189    465        3923     5
## 2 1956 Q2   213    5178    204    532        4436     6
## 3 1956 Q3   227    5297    208    561        4806     7
## 4 1956 Q4   308    5681    197    570        4418     6
## 5 1957 Q1   262    5577    187    529        4339     5
## 6 1957 Q2   228    5651    214    604        4811     7
\end{verbatim}

\begin{Shaded}
\begin{Highlighting}[]
\NormalTok{brick\_plot1}\OtherTok{\textless{}{-}}\NormalTok{aus\_production}\SpecialCharTok{\%\textgreater{}\%}
                \FunctionTok{autoplot}\NormalTok{(Bricks)}\SpecialCharTok{+}
                \FunctionTok{labs}\NormalTok{(}\AttributeTok{title=} \StringTok{"Bricks Production"}\NormalTok{)}\SpecialCharTok{+}
  \FunctionTok{annotate}\NormalTok{(}\StringTok{"text"}\NormalTok{, }\AttributeTok{x =} \ConstantTok{Inf}\NormalTok{, }\AttributeTok{y =} \SpecialCharTok{{-}}\ConstantTok{Inf}\NormalTok{, }\AttributeTok{hjust =} \DecValTok{1}\NormalTok{, }\AttributeTok{vjust =} \SpecialCharTok{{-}}\DecValTok{1}\NormalTok{,}
           \AttributeTok{label =} \StringTok{"There is obvious seasonality but not trend"}\NormalTok{)}

\NormalTok{brick\_plot2}\OtherTok{\textless{}{-}}\NormalTok{aus\_production}\SpecialCharTok{\%\textgreater{}\%}
\CommentTok{\# Warning: Removed 20 rows containing missing values (\textasciigrave{}geom\_line()\textasciigrave{}).}
\CommentTok{\# therefore filter is added}
  \FunctionTok{filter}\NormalTok{(}\SpecialCharTok{!}\FunctionTok{is.na}\NormalTok{(Bricks))}\SpecialCharTok{\%\textgreater{}\%}
  \FunctionTok{model}\NormalTok{(}\FunctionTok{SNAIVE}\NormalTok{(Bricks}\SpecialCharTok{\textasciitilde{}}\FunctionTok{lag}\NormalTok{(}\StringTok{"year"}\NormalTok{)))}\SpecialCharTok{\%\textgreater{}\%}
    \FunctionTok{forecast}\NormalTok{(}\AttributeTok{h =} \DecValTok{10}\NormalTok{)}\SpecialCharTok{\%\textgreater{}\%}
      \FunctionTok{autoplot}\NormalTok{(aus\_production)}\SpecialCharTok{+}
\CommentTok{\# aus\_production added twice to ensure visual has full plot}
  \FunctionTok{labs}\NormalTok{(}\AttributeTok{title=} \StringTok{"Bricks Production Forecast"}\NormalTok{)}

\FunctionTok{plot\_grid}\NormalTok{(brick\_plot1,brick\_plot2, }\AttributeTok{ncol =} \DecValTok{1}\NormalTok{)}
\end{Highlighting}
\end{Shaded}

\begin{verbatim}
## Warning: Removed 20 rows containing missing values (`geom_line()`).
## Removed 20 rows containing missing values (`geom_line()`).
\end{verbatim}

\includegraphics{Gabriel_Campos_DATA624_HW3_files/figure-latex/unnamed-chunk-6-1.pdf}
There was obvious seasonality despite the fact that annually there was
no trend. Regardless \texttt{SNAIVE(y)} was used as it seemed to be the
best fit.

\hypertarget{iii}{%
\subsection{iii}\label{iii}}

NSW Lambs (\texttt{aus\_livestock})

\begin{Shaded}
\begin{Highlighting}[]
\FunctionTok{cat}\NormalTok{(}\FunctionTok{paste}\NormalTok{(}\FunctionTok{unique}\NormalTok{(aus\_livestock}\SpecialCharTok{$}\NormalTok{State), }\AttributeTok{collapse =} \StringTok{"}\SpecialCharTok{\textbackslash{}n}\StringTok{"}\NormalTok{))}
\end{Highlighting}
\end{Shaded}

\begin{verbatim}
## Australian Capital Territory
## New South Wales
## Northern Territory
## Queensland
## South Australia
## Tasmania
## Victoria
## Western Australia
\end{verbatim}

\begin{Shaded}
\begin{Highlighting}[]
\NormalTok{df\_lambs}\OtherTok{\textless{}{-}}\NormalTok{aus\_livestock}\SpecialCharTok{\%\textgreater{}\%}
  \FunctionTok{filter}\NormalTok{(State }\SpecialCharTok{==} \StringTok{"New South Wales"}\NormalTok{, }\FunctionTok{str\_detect}\NormalTok{(Animal,}\StringTok{"Lambs"}\NormalTok{))}
\FunctionTok{head}\NormalTok{(df\_lambs)}
\end{Highlighting}
\end{Shaded}

\begin{verbatim}
## # A tsibble: 6 x 4 [1M]
## # Key:       Animal, State [1]
##      Month Animal State            Count
##      <mth> <fct>  <fct>            <dbl>
## 1 1972 Jul Lambs  New South Wales 587600
## 2 1972 Aug Lambs  New South Wales 553700
## 3 1972 Sep Lambs  New South Wales 494900
## 4 1972 Oct Lambs  New South Wales 533500
## 5 1972 Nov Lambs  New South Wales 574300
## 6 1972 Dec Lambs  New South Wales 517500
\end{verbatim}

\begin{Shaded}
\begin{Highlighting}[]
\NormalTok{lambs\_plot1 }\OtherTok{\textless{}{-}}\NormalTok{ df\_lambs}\SpecialCharTok{\%\textgreater{}\%}
                \FunctionTok{autoplot}\NormalTok{()}\SpecialCharTok{+}
                \FunctionTok{labs}\NormalTok{(}\AttributeTok{title=} \StringTok{"New South Wales Count"}\NormalTok{)}\SpecialCharTok{+}
  \FunctionTok{annotate}\NormalTok{(}\StringTok{"text"}\NormalTok{, }\AttributeTok{x =} \ConstantTok{Inf}\NormalTok{, }\AttributeTok{y =} \SpecialCharTok{{-}}\ConstantTok{Inf}\NormalTok{, }\AttributeTok{hjust =} \DecValTok{1}\NormalTok{, }\AttributeTok{vjust =} \SpecialCharTok{{-}}\DecValTok{1}\NormalTok{,}
           \AttributeTok{label =} \StringTok{"There is no obvious seasonality or trend"}\NormalTok{)}
\end{Highlighting}
\end{Shaded}

\begin{verbatim}
## Plot variable not specified, automatically selected `.vars = Count`
\end{verbatim}

\begin{Shaded}
\begin{Highlighting}[]
\NormalTok{lambs\_plot2}\OtherTok{\textless{}{-}}\NormalTok{df\_lambs}\SpecialCharTok{\%\textgreater{}\%}
   \FunctionTok{model}\NormalTok{(}\FunctionTok{NAIVE}\NormalTok{(Count))}\SpecialCharTok{\%\textgreater{}\%}
     \FunctionTok{forecast}\NormalTok{(}\AttributeTok{h =} \DecValTok{10}\NormalTok{)}\SpecialCharTok{\%\textgreater{}\%}
       \FunctionTok{autoplot}\NormalTok{(df\_lambs)}\SpecialCharTok{+}
   \FunctionTok{labs}\NormalTok{(}\AttributeTok{title=} \StringTok{"New South Wales Forecast"}\NormalTok{)}
\end{Highlighting}
\end{Shaded}

\includegraphics{Gabriel_Campos_DATA624_HW3_files/figure-latex/unnamed-chunk-10-1.pdf}

\includegraphics{Gabriel_Campos_DATA624_HW3_files/figure-latex/unnamed-chunk-11-1.pdf}

No clear seasonality or trend so \texttt{NAIVE(y)} was most appropriate

\hypertarget{iv}{%
\subsection{iv}\label{iv}}

Household wealth (\texttt{hh\_budget}).

\begin{Shaded}
\begin{Highlighting}[]
\FunctionTok{head}\NormalTok{(hh\_budget)}
\end{Highlighting}
\end{Shaded}

\begin{verbatim}
## # A tsibble: 6 x 8 [1Y]
## # Key:       Country [1]
##   Country    Year  Debt    DI Expenditure Savings Wealth Unemployment
##   <chr>     <dbl> <dbl> <dbl>       <dbl>   <dbl>  <dbl>        <dbl>
## 1 Australia  1995  95.7  3.72        3.40   5.24    315.         8.47
## 2 Australia  1996  99.5  3.98        2.97   6.47    315.         8.51
## 3 Australia  1997 108.   2.52        4.95   3.74    323.         8.36
## 4 Australia  1998 115.   4.02        5.73   1.29    339.         7.68
## 5 Australia  1999 121.   3.84        4.26   0.638   354.         6.87
## 6 Australia  2000 126.   3.77        3.18   1.99    350.         6.29
\end{verbatim}

\begin{Shaded}
\begin{Highlighting}[]
\NormalTok{wealth\_plot1}\OtherTok{\textless{}{-}}\NormalTok{hh\_budget}\SpecialCharTok{\%\textgreater{}\%}
                \FunctionTok{autoplot}\NormalTok{(Wealth, }\AttributeTok{show.legend=} \ConstantTok{FALSE}\NormalTok{)}\SpecialCharTok{+}
                  \FunctionTok{facet\_grid}\NormalTok{(Country}\SpecialCharTok{\textasciitilde{}}\NormalTok{., }\AttributeTok{scales =} \StringTok{"free"}\NormalTok{, }\AttributeTok{space =} \StringTok{"free\_y"}\NormalTok{)}

\NormalTok{wealth\_plot2}\OtherTok{\textless{}{-}}\NormalTok{ hh\_budget}\SpecialCharTok{\%\textgreater{}\%}
                \FunctionTok{model}\NormalTok{(}\FunctionTok{RW}\NormalTok{(Wealth}\SpecialCharTok{\textasciitilde{}}\FunctionTok{drift}\NormalTok{()))}\SpecialCharTok{\%\textgreater{}\%}
                  \FunctionTok{forecast}\NormalTok{(}\AttributeTok{h=}\DecValTok{5}\NormalTok{)}\SpecialCharTok{\%\textgreater{}\%}
                    \FunctionTok{autoplot}\NormalTok{(hh\_budget)}
\end{Highlighting}
\end{Shaded}

\includegraphics{Gabriel_Campos_DATA624_HW3_files/figure-latex/unnamed-chunk-14-1.pdf}

\includegraphics{Gabriel_Campos_DATA624_HW3_files/figure-latex/unnamed-chunk-15-1.pdf}

I genuinely considered the \texttt{NAIVE(y)} model, because the data was
a time series regarding finance. I also debated the seasonal aspect and
considered \texttt{SNAIVE(y)}, noting the dip in wealth at certain
intervals for certain countries. However, I think the trend is primarily
upward, with exception of years that tie in with the recent recessions,
therefore \texttt{RW(y\ \textasciitilde{}\ drift())} or Drift method was
used.

\hypertarget{v}{%
\subsection{v}\label{v}}

Australian takeaway food turnover (\texttt{aus\_retail}).

\begin{Shaded}
\begin{Highlighting}[]
\FunctionTok{cat}\NormalTok{(}\FunctionTok{paste}\NormalTok{(}\FunctionTok{unique}\NormalTok{(aus\_retail}\SpecialCharTok{$}\NormalTok{State), }\AttributeTok{collapse =} \StringTok{"}\SpecialCharTok{\textbackslash{}n}\StringTok{"}\NormalTok{))}
\end{Highlighting}
\end{Shaded}

\begin{verbatim}
## Australian Capital Territory
## New South Wales
## Northern Territory
## Queensland
## South Australia
## Tasmania
## Victoria
## Western Australia
\end{verbatim}

\begin{Shaded}
\begin{Highlighting}[]
\FunctionTok{head}\NormalTok{(aus\_retail)}
\end{Highlighting}
\end{Shaded}

\begin{verbatim}
## # A tsibble: 6 x 5 [1M]
## # Key:       State, Industry [1]
##   State                        Industry            `Series ID`    Month Turnover
##   <chr>                        <chr>               <chr>          <mth>    <dbl>
## 1 Australian Capital Territory Cafes, restaurants~ A3349849A   1982 Apr      4.4
## 2 Australian Capital Territory Cafes, restaurants~ A3349849A   1982 May      3.4
## 3 Australian Capital Territory Cafes, restaurants~ A3349849A   1982 Jun      3.6
## 4 Australian Capital Territory Cafes, restaurants~ A3349849A   1982 Jul      4  
## 5 Australian Capital Territory Cafes, restaurants~ A3349849A   1982 Aug      3.6
## 6 Australian Capital Territory Cafes, restaurants~ A3349849A   1982 Sep      4.2
\end{verbatim}

\begin{Shaded}
\begin{Highlighting}[]
\NormalTok{aus\_retail}\SpecialCharTok{\%\textgreater{}\%}
  \FunctionTok{filter}\NormalTok{(}\FunctionTok{str\_detect}\NormalTok{(Industry,}\StringTok{"takeaway"}\NormalTok{))}\SpecialCharTok{\%\textgreater{}\%}
  \FunctionTok{autoplot}\NormalTok{(Turnover)}\SpecialCharTok{+}
  \FunctionTok{scale\_color\_discrete}\NormalTok{(}\AttributeTok{name =} \StringTok{"State"}\NormalTok{, }\AttributeTok{labels =} \FunctionTok{unique}\NormalTok{(aus\_retail}\SpecialCharTok{$}\NormalTok{State))}\SpecialCharTok{+}
  \FunctionTok{labs}\NormalTok{(}\AttributeTok{title =} \StringTok{"Turnover (Australian takeaway) by State"}\NormalTok{)}
\end{Highlighting}
\end{Shaded}

\includegraphics{Gabriel_Campos_DATA624_HW3_files/figure-latex/unnamed-chunk-18-1.pdf}

\begin{Shaded}
\begin{Highlighting}[]
\NormalTok{aus\_retail }\SpecialCharTok{\%\textgreater{}\%}
  \FunctionTok{filter}\NormalTok{(}\FunctionTok{str\_detect}\NormalTok{(Industry,}\StringTok{"takeaway"}\NormalTok{)) }\SpecialCharTok{\%\textgreater{}\%}
  \FunctionTok{model}\NormalTok{(}\FunctionTok{RW}\NormalTok{(Turnover }\SpecialCharTok{\textasciitilde{}} \FunctionTok{drift}\NormalTok{())) }\SpecialCharTok{\%\textgreater{}\%}
  \FunctionTok{forecast}\NormalTok{(}\AttributeTok{h =} \DecValTok{10}\NormalTok{) }\SpecialCharTok{\%\textgreater{}\%}
  \FunctionTok{autoplot}\NormalTok{(aus\_retail)}\SpecialCharTok{+}
   \FunctionTok{facet\_wrap}\NormalTok{(}\SpecialCharTok{\textasciitilde{}}\NormalTok{State, }\AttributeTok{scales =} \StringTok{"free"}\NormalTok{)}
\end{Highlighting}
\end{Shaded}

\includegraphics{Gabriel_Campos_DATA624_HW3_files/figure-latex/unnamed-chunk-19-1.pdf}

Just like the

\begin{verbatim}
          Australian Population (`global_economy`)
          
\end{verbatim}

The data for all states showed an upward trend. So modeling each state
using the Drift method made the mose sense.

\hypertarget{section-1}{%
\section{5.2}\label{section-1}}

Use the Facebook stock price (data set \texttt{gafa\_stock}) to do the
following:

\hypertarget{a.}{%
\subsection{a.}\label{a.}}

Produce a time plot of the series.

\begin{Shaded}
\begin{Highlighting}[]
\FunctionTok{cat}\NormalTok{(}\FunctionTok{paste}\NormalTok{(}\FunctionTok{unique}\NormalTok{(gafa\_stock}\SpecialCharTok{$}\NormalTok{Symbol), }\AttributeTok{collapse =} \StringTok{"}\SpecialCharTok{\textbackslash{}n}\StringTok{"}\NormalTok{))}
\end{Highlighting}
\end{Shaded}

\begin{verbatim}
## AAPL
## AMZN
## FB
## GOOG
\end{verbatim}

\begin{Shaded}
\begin{Highlighting}[]
\FunctionTok{distinct}\NormalTok{(gafa\_stock, }\AttributeTok{year =}\NormalTok{ lubridate}\SpecialCharTok{::}\FunctionTok{year}\NormalTok{(Date))}
\end{Highlighting}
\end{Shaded}

\begin{verbatim}
## # A tibble: 5 x 1
##    year
##   <dbl>
## 1  2014
## 2  2015
## 3  2016
## 4  2017
## 5  2018
\end{verbatim}

\begin{Shaded}
\begin{Highlighting}[]
\NormalTok{df\_fb }\OtherTok{\textless{}{-}}\NormalTok{ gafa\_stock }\SpecialCharTok{\%\textgreater{}\%}
  \FunctionTok{filter}\NormalTok{(Symbol }\SpecialCharTok{==} \StringTok{"FB"}\NormalTok{)}

\FunctionTok{head}\NormalTok{(df\_fb)}
\end{Highlighting}
\end{Shaded}

\begin{verbatim}
## # A tsibble: 6 x 8 [!]
## # Key:       Symbol [1]
##   Symbol Date        Open  High   Low Close Adj_Close   Volume
##   <chr>  <date>     <dbl> <dbl> <dbl> <dbl>     <dbl>    <dbl>
## 1 FB     2014-01-02  54.8  55.2  54.2  54.7      54.7 43195500
## 2 FB     2014-01-03  55.0  55.7  54.5  54.6      54.6 38246200
## 3 FB     2014-01-06  54.4  57.3  54.0  57.2      57.2 68852600
## 4 FB     2014-01-07  57.7  58.5  57.2  57.9      57.9 77207400
## 5 FB     2014-01-08  57.6  58.4  57.2  58.2      58.2 56682400
## 6 FB     2014-01-09  58.7  59.0  56.7  57.2      57.2 92253300
\end{verbatim}

\begin{Shaded}
\begin{Highlighting}[]
\CommentTok{\# Re{-}index based on trading days}
\NormalTok{FB\_stock }\OtherTok{\textless{}{-}}\NormalTok{ df\_fb }\SpecialCharTok{\%\textgreater{}\%}
\CommentTok{\# already filtered}
  \FunctionTok{mutate}\NormalTok{(}\AttributeTok{day =} \FunctionTok{row\_number}\NormalTok{()) }\SpecialCharTok{\%\textgreater{}\%}
  \FunctionTok{update\_tsibble}\NormalTok{(}\AttributeTok{index =}\NormalTok{ day, }\AttributeTok{regular =} \ConstantTok{TRUE}\NormalTok{)}

\NormalTok{FB\_stock}\SpecialCharTok{\%\textgreater{}\%}
  \FunctionTok{autoplot}\NormalTok{(Close)}\SpecialCharTok{+}
  \FunctionTok{labs}\NormalTok{(}\AttributeTok{y =} \StringTok{\textquotesingle{}$US\textquotesingle{}}\NormalTok{, }\AttributeTok{title =} \StringTok{\textquotesingle{}The Facebook Daily Closing Stock Price\textquotesingle{}}\NormalTok{)}
\end{Highlighting}
\end{Shaded}

\includegraphics{Gabriel_Campos_DATA624_HW3_files/figure-latex/unnamed-chunk-23-1.pdf}

\hypertarget{b.}{%
\subsection{b.}\label{b.}}

Produce forecasts using the drift method and plot them.

\textbf{AS PER Example: Google's daily closing stock price}

\begin{Shaded}
\begin{Highlighting}[]
\CommentTok{\# Filter the year of interest}
\NormalTok{FB\_2015 }\OtherTok{\textless{}{-}}\NormalTok{ FB\_stock }\SpecialCharTok{\%\textgreater{}\%} \FunctionTok{filter}\NormalTok{(}\FunctionTok{year}\NormalTok{(Date) }\SpecialCharTok{==} \DecValTok{2015}\NormalTok{)}
\CommentTok{\# Fit the models}
\NormalTok{FB\_fit }\OtherTok{\textless{}{-}}\NormalTok{ FB\_2015 }\SpecialCharTok{|\textgreater{}}
  \FunctionTok{model}\NormalTok{(}
    \AttributeTok{Drift =} \FunctionTok{NAIVE}\NormalTok{(Close }\SpecialCharTok{\textasciitilde{}} \FunctionTok{drift}\NormalTok{())}
\NormalTok{  )}
\CommentTok{\# Produce forecasts for the trading days in January 2016}
\NormalTok{FB\_jan\_2016 }\OtherTok{\textless{}{-}}\NormalTok{ FB\_stock }\SpecialCharTok{|\textgreater{}}
  \FunctionTok{filter}\NormalTok{(}\FunctionTok{yearmonth}\NormalTok{(Date) }\SpecialCharTok{==} \FunctionTok{yearmonth}\NormalTok{(}\StringTok{"2016 Jan"}\NormalTok{))}
\NormalTok{FB\_fc }\OtherTok{\textless{}{-}}\NormalTok{ FB\_fit }\SpecialCharTok{|\textgreater{}}
  \FunctionTok{forecast}\NormalTok{(}\AttributeTok{new\_data =}\NormalTok{ FB\_jan\_2016)}
\CommentTok{\# Plot the forecasts}
\NormalTok{FB\_fc }\SpecialCharTok{|\textgreater{}}
  \FunctionTok{autoplot}\NormalTok{(FB\_2015, }\AttributeTok{level =} \ConstantTok{NULL}\NormalTok{) }\SpecialCharTok{+}
  \FunctionTok{autolayer}\NormalTok{(FB\_jan\_2016, Close, }\AttributeTok{colour =} \StringTok{"black"}\NormalTok{) }\SpecialCharTok{+}
  \FunctionTok{labs}\NormalTok{(}\AttributeTok{y =} \StringTok{"$US"}\NormalTok{,}
       \AttributeTok{title =} \StringTok{"Facebook daily closing stock prices"}\NormalTok{,}
       \AttributeTok{subtitle =} \StringTok{"(Jan 2015 {-} Jan 2016)"}\NormalTok{) }\SpecialCharTok{+}
  \FunctionTok{guides}\NormalTok{(}\AttributeTok{colour =} \FunctionTok{guide\_legend}\NormalTok{(}\AttributeTok{title =} \StringTok{"Forecast"}\NormalTok{))}
\end{Highlighting}
\end{Shaded}

\includegraphics{Gabriel_Campos_DATA624_HW3_files/figure-latex/unnamed-chunk-24-1.pdf}

\hypertarget{c.}{%
\subsection{c.}\label{c.}}

Show that the forecasts are identical to extending the line drawn
between the first and last observations.

\begin{Shaded}
\begin{Highlighting}[]
\NormalTok{FB\_fc}\SpecialCharTok{\%\textgreater{}\%} 
  \FunctionTok{autoplot}\NormalTok{(FB\_2015, }\AttributeTok{level =} \ConstantTok{NULL}\NormalTok{) }\SpecialCharTok{+}
  \FunctionTok{geom\_line}\NormalTok{(}\AttributeTok{data =} \FunctionTok{slice}\NormalTok{(FB\_2015, }\FunctionTok{range}\NormalTok{(}\FunctionTok{cumsum}\NormalTok{(}\SpecialCharTok{!}\FunctionTok{is.na}\NormalTok{(Close)))),}
                         \FunctionTok{aes}\NormalTok{(}\AttributeTok{y=}\NormalTok{Close), }\AttributeTok{linetype =} \StringTok{\textquotesingle{}dashed\textquotesingle{}}\NormalTok{)}
\end{Highlighting}
\end{Shaded}

\includegraphics{Gabriel_Campos_DATA624_HW3_files/figure-latex/unnamed-chunk-25-1.pdf}

\hypertarget{d.}{%
\subsection{d.}\label{d.}}

Try using some of the other benchmark functions to forecast the same
data set. Which do you think is best? Why?

\begin{Shaded}
\begin{Highlighting}[]
\NormalTok{FB\_fit2 }\OtherTok{\textless{}{-}}\NormalTok{ FB\_2015 }\SpecialCharTok{\%\textgreater{}\%}
  \FunctionTok{model}\NormalTok{(}
    \AttributeTok{Mean =} \FunctionTok{MEAN}\NormalTok{(Close),}
    \AttributeTok{Naive =} \FunctionTok{NAIVE}\NormalTok{(Close)}
\NormalTok{  )}
\CommentTok{\# to make the forecasts for the trading days in January 2016}
\NormalTok{FB\_jan\_2016 }\OtherTok{\textless{}{-}}\NormalTok{ FB\_stock }\SpecialCharTok{\%\textgreater{}\%}
  \FunctionTok{filter}\NormalTok{(}\FunctionTok{yearmonth}\NormalTok{(Date) }\SpecialCharTok{==} \FunctionTok{yearmonth}\NormalTok{(}\StringTok{"2016 Jan"}\NormalTok{))}

\NormalTok{FB\_fc2 }\OtherTok{\textless{}{-}}\NormalTok{ FB\_fit2 }\SpecialCharTok{\%\textgreater{}\%}
  \FunctionTok{forecast}\NormalTok{(}\AttributeTok{new\_data =}\NormalTok{ FB\_jan\_2016)}
\CommentTok{\# Plotting}
\NormalTok{FB\_fc2 }\SpecialCharTok{\%\textgreater{}\%}
  \FunctionTok{autoplot}\NormalTok{(FB\_2015, }\AttributeTok{level =} \ConstantTok{NULL}\NormalTok{) }\SpecialCharTok{+}
  \FunctionTok{autolayer}\NormalTok{(FB\_jan\_2016, Close, }\AttributeTok{colour =} \StringTok{"green"}\NormalTok{) }\SpecialCharTok{+}
  \FunctionTok{labs}\NormalTok{(}\AttributeTok{y =} \StringTok{"$USD"}\NormalTok{,}
       \AttributeTok{title =} \StringTok{"FB  Closing Stock Prices (Daily)"}\NormalTok{,}
       \AttributeTok{subtitle =} \StringTok{"(Jan 2015 {-} Jan 2016)"}\NormalTok{) }\SpecialCharTok{+}
  \FunctionTok{guides}\NormalTok{(}\AttributeTok{colour =} \FunctionTok{guide\_legend}\NormalTok{(}\AttributeTok{title =} \StringTok{"The Forecast"}\NormalTok{))}
\end{Highlighting}
\end{Shaded}

\includegraphics{Gabriel_Campos_DATA624_HW3_files/figure-latex/unnamed-chunk-26-1.pdf}

Naive I believe is the most accurate, b/c there is no seasonality and
Naive works best with financial data according to the textbook

\begin{verbatim}
        "This method works remarkably well for many economic and financial time series."
\end{verbatim}

\end{document}
