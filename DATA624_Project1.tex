% Options for packages loaded elsewhere
\PassOptionsToPackage{unicode}{hyperref}
\PassOptionsToPackage{hyphens}{url}
\PassOptionsToPackage{dvipsnames,svgnames,x11names}{xcolor}
%
\documentclass[
]{article}
\usepackage{amsmath,amssymb}
\usepackage{iftex}
\ifPDFTeX
  \usepackage[T1]{fontenc}
  \usepackage[utf8]{inputenc}
  \usepackage{textcomp} % provide euro and other symbols
\else % if luatex or xetex
  \usepackage{unicode-math} % this also loads fontspec
  \defaultfontfeatures{Scale=MatchLowercase}
  \defaultfontfeatures[\rmfamily]{Ligatures=TeX,Scale=1}
\fi
\usepackage{lmodern}
\ifPDFTeX\else
  % xetex/luatex font selection
\fi
% Use upquote if available, for straight quotes in verbatim environments
\IfFileExists{upquote.sty}{\usepackage{upquote}}{}
\IfFileExists{microtype.sty}{% use microtype if available
  \usepackage[]{microtype}
  \UseMicrotypeSet[protrusion]{basicmath} % disable protrusion for tt fonts
}{}
\makeatletter
\@ifundefined{KOMAClassName}{% if non-KOMA class
  \IfFileExists{parskip.sty}{%
    \usepackage{parskip}
  }{% else
    \setlength{\parindent}{0pt}
    \setlength{\parskip}{6pt plus 2pt minus 1pt}}
}{% if KOMA class
  \KOMAoptions{parskip=half}}
\makeatother
\usepackage{xcolor}
\usepackage[margin=1in]{geometry}
\usepackage{color}
\usepackage{fancyvrb}
\newcommand{\VerbBar}{|}
\newcommand{\VERB}{\Verb[commandchars=\\\{\}]}
\DefineVerbatimEnvironment{Highlighting}{Verbatim}{commandchars=\\\{\}}
% Add ',fontsize=\small' for more characters per line
\usepackage{framed}
\definecolor{shadecolor}{RGB}{248,248,248}
\newenvironment{Shaded}{\begin{snugshade}}{\end{snugshade}}
\newcommand{\AlertTok}[1]{\textcolor[rgb]{0.94,0.16,0.16}{#1}}
\newcommand{\AnnotationTok}[1]{\textcolor[rgb]{0.56,0.35,0.01}{\textbf{\textit{#1}}}}
\newcommand{\AttributeTok}[1]{\textcolor[rgb]{0.13,0.29,0.53}{#1}}
\newcommand{\BaseNTok}[1]{\textcolor[rgb]{0.00,0.00,0.81}{#1}}
\newcommand{\BuiltInTok}[1]{#1}
\newcommand{\CharTok}[1]{\textcolor[rgb]{0.31,0.60,0.02}{#1}}
\newcommand{\CommentTok}[1]{\textcolor[rgb]{0.56,0.35,0.01}{\textit{#1}}}
\newcommand{\CommentVarTok}[1]{\textcolor[rgb]{0.56,0.35,0.01}{\textbf{\textit{#1}}}}
\newcommand{\ConstantTok}[1]{\textcolor[rgb]{0.56,0.35,0.01}{#1}}
\newcommand{\ControlFlowTok}[1]{\textcolor[rgb]{0.13,0.29,0.53}{\textbf{#1}}}
\newcommand{\DataTypeTok}[1]{\textcolor[rgb]{0.13,0.29,0.53}{#1}}
\newcommand{\DecValTok}[1]{\textcolor[rgb]{0.00,0.00,0.81}{#1}}
\newcommand{\DocumentationTok}[1]{\textcolor[rgb]{0.56,0.35,0.01}{\textbf{\textit{#1}}}}
\newcommand{\ErrorTok}[1]{\textcolor[rgb]{0.64,0.00,0.00}{\textbf{#1}}}
\newcommand{\ExtensionTok}[1]{#1}
\newcommand{\FloatTok}[1]{\textcolor[rgb]{0.00,0.00,0.81}{#1}}
\newcommand{\FunctionTok}[1]{\textcolor[rgb]{0.13,0.29,0.53}{\textbf{#1}}}
\newcommand{\ImportTok}[1]{#1}
\newcommand{\InformationTok}[1]{\textcolor[rgb]{0.56,0.35,0.01}{\textbf{\textit{#1}}}}
\newcommand{\KeywordTok}[1]{\textcolor[rgb]{0.13,0.29,0.53}{\textbf{#1}}}
\newcommand{\NormalTok}[1]{#1}
\newcommand{\OperatorTok}[1]{\textcolor[rgb]{0.81,0.36,0.00}{\textbf{#1}}}
\newcommand{\OtherTok}[1]{\textcolor[rgb]{0.56,0.35,0.01}{#1}}
\newcommand{\PreprocessorTok}[1]{\textcolor[rgb]{0.56,0.35,0.01}{\textit{#1}}}
\newcommand{\RegionMarkerTok}[1]{#1}
\newcommand{\SpecialCharTok}[1]{\textcolor[rgb]{0.81,0.36,0.00}{\textbf{#1}}}
\newcommand{\SpecialStringTok}[1]{\textcolor[rgb]{0.31,0.60,0.02}{#1}}
\newcommand{\StringTok}[1]{\textcolor[rgb]{0.31,0.60,0.02}{#1}}
\newcommand{\VariableTok}[1]{\textcolor[rgb]{0.00,0.00,0.00}{#1}}
\newcommand{\VerbatimStringTok}[1]{\textcolor[rgb]{0.31,0.60,0.02}{#1}}
\newcommand{\WarningTok}[1]{\textcolor[rgb]{0.56,0.35,0.01}{\textbf{\textit{#1}}}}
\usepackage{graphicx}
\makeatletter
\def\maxwidth{\ifdim\Gin@nat@width>\linewidth\linewidth\else\Gin@nat@width\fi}
\def\maxheight{\ifdim\Gin@nat@height>\textheight\textheight\else\Gin@nat@height\fi}
\makeatother
% Scale images if necessary, so that they will not overflow the page
% margins by default, and it is still possible to overwrite the defaults
% using explicit options in \includegraphics[width, height, ...]{}
\setkeys{Gin}{width=\maxwidth,height=\maxheight,keepaspectratio}
% Set default figure placement to htbp
\makeatletter
\def\fps@figure{htbp}
\makeatother
\setlength{\emergencystretch}{3em} % prevent overfull lines
\providecommand{\tightlist}{%
  \setlength{\itemsep}{0pt}\setlength{\parskip}{0pt}}
\setcounter{secnumdepth}{-\maxdimen} % remove section numbering
\ifLuaTeX
  \usepackage{selnolig}  % disable illegal ligatures
\fi
\IfFileExists{bookmark.sty}{\usepackage{bookmark}}{\usepackage{hyperref}}
\IfFileExists{xurl.sty}{\usepackage{xurl}}{} % add URL line breaks if available
\urlstyle{same}
\hypersetup{
  pdftitle={DATA 624: PREDICTIVE ANALYTICS Project 1},
  pdfauthor={Gabriel Campos},
  colorlinks=true,
  linkcolor={Maroon},
  filecolor={Maroon},
  citecolor={Blue},
  urlcolor={blue},
  pdfcreator={LaTeX via pandoc}}

\title{DATA 624: PREDICTIVE ANALYTICS Project 1}
\author{Gabriel Campos}
\date{Last edited March 23, 2024}

\begin{document}
\maketitle

\begin{Shaded}
\begin{Highlighting}[]
\FunctionTok{library}\NormalTok{(fpp3)}
\FunctionTok{library}\NormalTok{(dplyr)}
\FunctionTok{library}\NormalTok{(ggplot2)}
\FunctionTok{library}\NormalTok{(readxl)}
\end{Highlighting}
\end{Shaded}

\begin{verbatim}
## Warning: package 'readxl' was built under R version 4.3.3
\end{verbatim}

\begin{Shaded}
\begin{Highlighting}[]
\FunctionTok{library}\NormalTok{(tsibble)}
\FunctionTok{library}\NormalTok{(psych)}
\FunctionTok{library}\NormalTok{(tidyr)}
\FunctionTok{library}\NormalTok{(forecast)}
\end{Highlighting}
\end{Shaded}

\begin{verbatim}
## Warning: package 'forecast' was built under R version 4.3.3
\end{verbatim}

\hypertarget{description}{%
\section{Description}\label{description}}

This project consists of 3 parts - two required and one bonus and is
worth 15\% of your grade. The project is due at 11:59 PM on Sunday Apr
11. I will accept late submissions with a penalty until the meetup after
that when we review some projects.

\hypertarget{part-a}{%
\subsection{Part A}\label{part-a}}

\textbf{ATM Forecast}
\href{https://bbhosted.cuny.edu/bbcswebdav/pid-81630946-dt-content-rid-636012399_1/xid-636012399_1}{ATM624Data.xlsx}

In part A, I want you to forecast how much cash is taken out of 4
different ATM machines for May 2010. The data is given in a single file.
The variable `Cash' is provided in hundreds of dollars, other than that
it is straight forward. I am being somewhat ambiguous on purpose to make
this have a little more business feeling. Explain and demonstrate your
process, techniques used and not used, and your actual forecast. I am
giving you data via an excel file, please provide your written report on
your findings, visuals, discussion and your R code via an RPubs link
along with the actual.rmd file Also please submit the forecast which you
will put in an Excel readable file.

\hypertarget{part-b}{%
\subsection{Part B}\label{part-b}}

Forecasting Power
\href{https://bbhosted.cuny.edu/bbcswebdav/pid-81630947-dt-content-rid-636015207_1/xid-636015207_1}{ResidentialCustomerForecastLoad-624.xlsx}

Part B consists of a simple dataset of residential power usage for
January 1998 until December 2013. Your assignment is to model these data
and a monthly forecast for 2014. The data is given in a single file. The
variable `KWH' is power consumption in Kilowatt hours, the rest is
straight forward. Add this to your existing files above.

\hypertarget{part-c}{%
\subsection{Part C}\label{part-c}}

BONUS, optional (part or all),
\href{https://bbhosted.cuny.edu/bbcswebdav/pid-81630948-dt-content-rid-636015213_1/xid-636015213_1}{Waterflow\_Pipe1.xlsx}
and
\href{https://bbhosted.cuny.edu/bbcswebdav/pid-81630949-dt-content-rid-636015214_1/xid-636015214_1}{Waterflow\_Pipe2.xlsx}

Part C consists of two data sets. These are simple 2 columns sets,
however they have different time stamps. Your optional assignment is to
time-base sequence the data and aggregate based on hour (example of what
this looks like, follows). Note for multiple recordings within an hour,
take the mean. Then to determine if the data is stationary and can it be
forecast. If so, provide a week forward forecast and present results via
Rpubs and .rmd and the forecast in an Excel readable file.

\hypertarget{data-load}{%
\subsection{Data Load}\label{data-load}}

\url{https://github.com/GitableGabe/Data624_Data/raw/main/ATM624Data.xlsx}

\begin{Shaded}
\begin{Highlighting}[]
\NormalTok{atm\_coltype}\OtherTok{\textless{}{-}}\FunctionTok{c}\NormalTok{(}\StringTok{"date"}\NormalTok{,}\StringTok{"text"}\NormalTok{,}\StringTok{"numeric"}\NormalTok{)}

\NormalTok{atm\_import}\OtherTok{\textless{}{-}}\FunctionTok{read\_xlsx}\NormalTok{(}\StringTok{\textquotesingle{}ATM624Data.xlsx\textquotesingle{}}\NormalTok{, }\AttributeTok{col\_types =}\NormalTok{ atm\_coltype)}
\CommentTok{\# Ommitting Extra Credit as I won\textquotesingle{}t be working on it}
\CommentTok{\# WP1\_df\textless{}{-}read\_xlsx(\textquotesingle{}Waterflow\_Pipe1.xlsx\textquotesingle{})}
\CommentTok{\# WP2\_df\textless{}{-}read\_xlsx(\textquotesingle{}Waterflow\_Pipe2.xlsx\textquotesingle{})}
\end{Highlighting}
\end{Shaded}

\begin{Shaded}
\begin{Highlighting}[]
\NormalTok{power\_raw}\OtherTok{\textless{}{-}}\FunctionTok{read\_xlsx}\NormalTok{(}\StringTok{\textquotesingle{}ResidentialCustomerForecastLoad{-}624.xlsx\textquotesingle{}}\NormalTok{)}
\end{Highlighting}
\end{Shaded}

\hypertarget{part-a-1}{%
\section{Part A}\label{part-a-1}}

\hypertarget{eda-cleanup}{%
\subsection{EDA \& Cleanup}\label{eda-cleanup}}

\begin{Shaded}
\begin{Highlighting}[]
\FunctionTok{head}\NormalTok{(atm\_import}\SpecialCharTok{\%\textgreater{}\%}
       \FunctionTok{filter}\NormalTok{(ATM}\SpecialCharTok{==}\StringTok{"ATM4"}\NormalTok{))}
\end{Highlighting}
\end{Shaded}

\begin{verbatim}
## # A tibble: 6 x 3
##   DATE                ATM    Cash
##   <dttm>              <chr> <dbl>
## 1 2009-05-01 00:00:00 ATM4  777. 
## 2 2009-05-02 00:00:00 ATM4  524. 
## 3 2009-05-03 00:00:00 ATM4  793. 
## 4 2009-05-04 00:00:00 ATM4  908. 
## 5 2009-05-05 00:00:00 ATM4   52.8
## 6 2009-05-06 00:00:00 ATM4   52.2
\end{verbatim}

\begin{Shaded}
\begin{Highlighting}[]
\NormalTok{atm\_range}\OtherTok{\textless{}{-}}\FunctionTok{range}\NormalTok{(atm\_import}\SpecialCharTok{$}\NormalTok{DATE)}
\NormalTok{atm\_range[}\DecValTok{1}\NormalTok{]}
\end{Highlighting}
\end{Shaded}

\begin{verbatim}
## [1] "2009-05-01 UTC"
\end{verbatim}

\begin{Shaded}
\begin{Highlighting}[]
\NormalTok{atm\_range[}\DecValTok{2}\NormalTok{]}
\end{Highlighting}
\end{Shaded}

\begin{verbatim}
## [1] "2010-05-14 UTC"
\end{verbatim}

\begin{Shaded}
\begin{Highlighting}[]
\FunctionTok{sapply}\NormalTok{(atm\_import, }\ControlFlowTok{function}\NormalTok{(x) }\FunctionTok{sum}\NormalTok{(}\FunctionTok{is.na}\NormalTok{(x)))}
\end{Highlighting}
\end{Shaded}

\begin{verbatim}
## DATE  ATM Cash 
##    0   14   19
\end{verbatim}

\begin{Shaded}
\begin{Highlighting}[]
\FunctionTok{data.frame}\NormalTok{(atm\_import}\SpecialCharTok{$}\NormalTok{DATE[atm\_import}\SpecialCharTok{$}\NormalTok{Cash }\SpecialCharTok{\%in\%} \ConstantTok{NA}\NormalTok{])}
\end{Highlighting}
\end{Shaded}

\begin{verbatim}
##    atm_import.DATE.atm_import.Cash..in..NA.
## 1                                2009-06-13
## 2                                2009-06-16
## 3                                2009-06-18
## 4                                2009-06-22
## 5                                2009-06-24
## 6                                2010-05-01
## 7                                2010-05-02
## 8                                2010-05-03
## 9                                2010-05-04
## 10                               2010-05-05
## 11                               2010-05-06
## 12                               2010-05-07
## 13                               2010-05-08
## 14                               2010-05-09
## 15                               2010-05-10
## 16                               2010-05-11
## 17                               2010-05-12
## 18                               2010-05-13
## 19                               2010-05-14
\end{verbatim}

\begin{itemize}
\tightlist
\item
  ATM624Data had attribute type mismatches, and was converted on import.
\item
  Date conversion somehow kept date time as POSIXct
\item
  ATM4 shows values in greater decimals any country, with Dinars being
  the only Country that uses more than 2 decimals when using its
  currency, but even the dinar stops at the 100th decimal.
\item
  Date range is 05-01-2009 to 05-14-2010
\item
  we see the count of NAs in ATM is 14 and Cash column is 19
\item
  The NA dates vary and are not exclusive to a specific sequential time
  period that we can just filter out.
\item
  I am curious about the distribution of cash considering the forecast
  ask for this project.
\end{itemize}

\begin{Shaded}
\begin{Highlighting}[]
\NormalTok{atm\_import }\SpecialCharTok{\%\textgreater{}\%} 
  \FunctionTok{filter}\NormalTok{(DATE }\SpecialCharTok{\textless{}} \StringTok{"2010{-}05{-}01"}\NormalTok{, }\SpecialCharTok{!}\FunctionTok{is.na}\NormalTok{(ATM)) }\SpecialCharTok{\%\textgreater{}\%} 
  \FunctionTok{ggplot}\NormalTok{(}\FunctionTok{aes}\NormalTok{(}\AttributeTok{x =}\NormalTok{ Cash)) }\SpecialCharTok{+}
    \FunctionTok{geom\_histogram}\NormalTok{(}\AttributeTok{bins =} \DecValTok{30}\NormalTok{, }\AttributeTok{color=} \StringTok{"blue"}\NormalTok{) }\SpecialCharTok{+}
    \FunctionTok{facet\_wrap}\NormalTok{(}\SpecialCharTok{\textasciitilde{}}\NormalTok{ ATM, }\AttributeTok{ncol =} \DecValTok{2}\NormalTok{, }\AttributeTok{scales =} \StringTok{"free"}\NormalTok{)}
\end{Highlighting}
\end{Shaded}

\begin{verbatim}
## Warning: Removed 5 rows containing non-finite values (`stat_bin()`).
\end{verbatim}

\includegraphics{DATA624_Project1_files/figure-latex/unnamed-chunk-8-1.pdf}

\begin{Shaded}
\begin{Highlighting}[]
\NormalTok{(atm\_df }\OtherTok{\textless{}{-}}\NormalTok{ atm\_import }\SpecialCharTok{\%\textgreater{}\%} 
  \FunctionTok{mutate}\NormalTok{(}\AttributeTok{DATE =} \FunctionTok{as.Date}\NormalTok{(DATE)) }\SpecialCharTok{\%\textgreater{}\%}
   \FunctionTok{filter}\NormalTok{(DATE}\SpecialCharTok{\textless{}}\StringTok{"2010{-}05{-}01"}\NormalTok{)}\SpecialCharTok{\%\textgreater{}\%}
  \FunctionTok{pivot\_wider}\NormalTok{(}\AttributeTok{names\_from=}\NormalTok{ATM, }\AttributeTok{values\_from =}\NormalTok{ Cash))}
\end{Highlighting}
\end{Shaded}

\begin{verbatim}
## # A tibble: 365 x 5
##    DATE        ATM1  ATM2  ATM3  ATM4
##    <date>     <dbl> <dbl> <dbl> <dbl>
##  1 2009-05-01    96   107     0 777. 
##  2 2009-05-02    82    89     0 524. 
##  3 2009-05-03    85    90     0 793. 
##  4 2009-05-04    90    55     0 908. 
##  5 2009-05-05    99    79     0  52.8
##  6 2009-05-06    88    19     0  52.2
##  7 2009-05-07     8     2     0  55.5
##  8 2009-05-08   104   103     0 559. 
##  9 2009-05-09    87   107     0 904. 
## 10 2009-05-10    93   118     0 879. 
## # i 355 more rows
\end{verbatim}

\begin{Shaded}
\begin{Highlighting}[]
\NormalTok{atm\_df}\OtherTok{\textless{}{-}}\NormalTok{atm\_df}\SpecialCharTok{\%\textgreater{}\%}
  \FunctionTok{as\_tsibble}\NormalTok{(}\AttributeTok{index=}\NormalTok{DATE)}
\FunctionTok{head}\NormalTok{(atm\_df)}
\end{Highlighting}
\end{Shaded}

\begin{verbatim}
## # A tsibble: 6 x 5 [1D]
##   DATE        ATM1  ATM2  ATM3  ATM4
##   <date>     <dbl> <dbl> <dbl> <dbl>
## 1 2009-05-01    96   107     0 777. 
## 2 2009-05-02    82    89     0 524. 
## 3 2009-05-03    85    90     0 793. 
## 4 2009-05-04    90    55     0 908. 
## 5 2009-05-05    99    79     0  52.8
## 6 2009-05-06    88    19     0  52.2
\end{verbatim}

\begin{Shaded}
\begin{Highlighting}[]
\FunctionTok{summary}\NormalTok{(atm\_df)}
\end{Highlighting}
\end{Shaded}

\begin{verbatim}
##       DATE                 ATM1             ATM2             ATM3        
##  Min.   :2009-05-01   Min.   :  1.00   Min.   :  0.00   Min.   : 0.0000  
##  1st Qu.:2009-07-31   1st Qu.: 73.00   1st Qu.: 25.50   1st Qu.: 0.0000  
##  Median :2009-10-30   Median : 91.00   Median : 67.00   Median : 0.0000  
##  Mean   :2009-10-30   Mean   : 83.89   Mean   : 62.58   Mean   : 0.7206  
##  3rd Qu.:2010-01-29   3rd Qu.:108.00   3rd Qu.: 93.00   3rd Qu.: 0.0000  
##  Max.   :2010-04-30   Max.   :180.00   Max.   :147.00   Max.   :96.0000  
##                       NA's   :3        NA's   :2                         
##       ATM4          
##  Min.   :    1.563  
##  1st Qu.:  124.334  
##  Median :  403.839  
##  Mean   :  474.043  
##  3rd Qu.:  704.507  
##  Max.   :10919.762  
## 
\end{verbatim}

\begin{Shaded}
\begin{Highlighting}[]
\NormalTok{atm\_df[}\SpecialCharTok{!}\FunctionTok{complete.cases}\NormalTok{(atm\_df), ]}
\end{Highlighting}
\end{Shaded}

\begin{verbatim}
## # A tsibble: 5 x 5 [1D]
##   DATE        ATM1  ATM2  ATM3  ATM4
##   <date>     <dbl> <dbl> <dbl> <dbl>
## 1 2009-06-13    NA    91     0 746. 
## 2 2009-06-16    NA    82     0 373. 
## 3 2009-06-18    21    NA     0  92.5
## 4 2009-06-22    NA    90     0  80.6
## 5 2009-06-24    66    NA     0  90.6
\end{verbatim}

\begin{Shaded}
\begin{Highlighting}[]
\NormalTok{atm\_df}\SpecialCharTok{\%\textgreater{}\%}
  \FunctionTok{select}\NormalTok{(DATE,ATM3)}\SpecialCharTok{\%\textgreater{}\%}
  \FunctionTok{filter}\NormalTok{(ATM3}\SpecialCharTok{\textgreater{}}\DecValTok{0}\NormalTok{)}
\end{Highlighting}
\end{Shaded}

\begin{verbatim}
## # A tsibble: 3 x 2 [1D]
##   DATE        ATM3
##   <date>     <dbl>
## 1 2010-04-28    96
## 2 2010-04-29    82
## 3 2010-04-30    85
\end{verbatim}

\begin{itemize}
\tightlist
\item
  Converting \texttt{DATE} into a date value made senses type POSIXct
  may cause future issues.
\item
  Pivoting allowed us to separate the ATM's categorically and isolate
  the NAs for removal.
\item
  We are able to see that five entries contain NAs and the dates all
  reside in June
\item
  ATM3 only has 3 dates with withdrawals 4-28 through 4-30 or 2010, and
  the distribution plot is arguably a reason to omit this column
\item
  These results also brings to question whether there may be some
  seasonality that will impact May's forecasting
\item
  Considering the distribution, I chose to replace the missing values
  with the median, as the skewed values in ATM 3 \& 4 I believe with
  negatively impact the mean
\end{itemize}

\begin{Shaded}
\begin{Highlighting}[]
\CommentTok{\# seasonality}
\NormalTok{atm\_import }\SpecialCharTok{\%\textgreater{}\%} 
  \FunctionTok{filter}\NormalTok{(DATE }\SpecialCharTok{\textless{}} \StringTok{"2010{-}05{-}01"}\NormalTok{, }\SpecialCharTok{!}\FunctionTok{is.na}\NormalTok{(ATM)) }\SpecialCharTok{\%\textgreater{}\%} 
  \FunctionTok{ggplot}\NormalTok{(}\FunctionTok{aes}\NormalTok{(}\AttributeTok{x =}\NormalTok{ DATE, }\AttributeTok{y =}\NormalTok{ Cash, }\AttributeTok{col =}\NormalTok{ ATM)) }\SpecialCharTok{+}
    \FunctionTok{geom\_line}\NormalTok{(}\AttributeTok{color=}\StringTok{"blue"}\NormalTok{) }\SpecialCharTok{+}
    \FunctionTok{facet\_wrap}\NormalTok{(}\SpecialCharTok{\textasciitilde{}}\NormalTok{ ATM, }\AttributeTok{ncol =} \DecValTok{2}\NormalTok{, }\AttributeTok{scales =} \StringTok{"free\_y"}\NormalTok{)}\SpecialCharTok{+}
  \FunctionTok{labs}\NormalTok{(}\AttributeTok{title =} \StringTok{"Seasonality Plot"}\NormalTok{, }\AttributeTok{x =} \StringTok{"Date"}\NormalTok{, }\AttributeTok{y =} \StringTok{"Cash"}\NormalTok{) }\SpecialCharTok{+}
    \FunctionTok{theme\_minimal}\NormalTok{()}
\end{Highlighting}
\end{Shaded}

\includegraphics{DATA624_Project1_files/figure-latex/unnamed-chunk-14-1.pdf}

\begin{Shaded}
\begin{Highlighting}[]
\NormalTok{median\_value }\OtherTok{\textless{}{-}} \FunctionTok{median}\NormalTok{(atm\_df[[}\StringTok{"ATM1"}\NormalTok{]], }\AttributeTok{na.rm =} \ConstantTok{TRUE}\NormalTok{)}
\NormalTok{atm\_df[[}\StringTok{"ATM1"}\NormalTok{]][}\FunctionTok{is.na}\NormalTok{(atm\_df[[}\StringTok{"ATM1"}\NormalTok{]])] }\OtherTok{\textless{}{-}}\NormalTok{ median\_value}
\NormalTok{median\_value }\OtherTok{\textless{}{-}} \FunctionTok{median}\NormalTok{(atm\_df[[}\StringTok{"ATM2"}\NormalTok{]], }\AttributeTok{na.rm =} \ConstantTok{TRUE}\NormalTok{)}
\NormalTok{atm\_df[[}\StringTok{"ATM2"}\NormalTok{]][}\FunctionTok{is.na}\NormalTok{(atm\_df[[}\StringTok{"ATM2"}\NormalTok{]])] }\OtherTok{\textless{}{-}}\NormalTok{ median\_value}
\end{Highlighting}
\end{Shaded}

\begin{Shaded}
\begin{Highlighting}[]
\NormalTok{atm\_df[}\SpecialCharTok{!}\FunctionTok{complete.cases}\NormalTok{(atm\_df), ]}
\end{Highlighting}
\end{Shaded}

\begin{verbatim}
## # A tsibble: 0 x 5 [?]
## # i 5 variables: DATE <date>, ATM1 <dbl>, ATM2 <dbl>, ATM3 <dbl>, ATM4 <dbl>
\end{verbatim}

\hypertarget{forecasts}{%
\subsection{Forecasts}\label{forecasts}}

\hypertarget{atm1}{%
\subsubsection{ATM1}\label{atm1}}

\hypertarget{stl-decomposition}{%
\paragraph{STL Decomposition}\label{stl-decomposition}}

The seasonality plot did not show a trend in the long term but a better
assessment in weekly interval is likely needed, using resources from
\href{https://otexts.com/fpp3/stl.html}{Rob J Hyndman and George
Athanasopoulos, Forecasting: Principles and Practice (3rd ed) section
3.6 STL decomposition} I will perform a STL ``Seasonal and Trend
decomposition using Loess'' decomposition of the series. To make it
weekly I'll set the parameter \texttt{trend(window\ =\ 7)} and the
\texttt{season(window=\textquotesingle{}periodic\textquotesingle{})} to
impose seasonality element across days of the week.

My reference come directly from the chapter.

\begin{verbatim}
          us_retail_employment |>
            model(
              STL(Employed ~ trend(window = 7) +
                             season(window = "periodic"),
              robust = TRUE)) |>
            components() |>
            autoplot()
\end{verbatim}

\begin{Shaded}
\begin{Highlighting}[]
\NormalTok{atm1\_df }\OtherTok{\textless{}{-}}\NormalTok{ atm\_df }\SpecialCharTok{\%\textgreater{}\%} 
\NormalTok{  dplyr}\SpecialCharTok{::}\FunctionTok{select}\NormalTok{(DATE, ATM1)}

\NormalTok{atm1\_df }\SpecialCharTok{\%\textgreater{}\%}
  \FunctionTok{model}\NormalTok{(}
    \FunctionTok{STL}\NormalTok{(ATM1 }\SpecialCharTok{\textasciitilde{}} \FunctionTok{trend}\NormalTok{(}\AttributeTok{window =} \DecValTok{7}\NormalTok{) }\SpecialCharTok{+}
                   \FunctionTok{season}\NormalTok{(}\AttributeTok{window =} \StringTok{"periodic"}\NormalTok{),}
    \AttributeTok{robust =} \ConstantTok{TRUE}\NormalTok{)) }\SpecialCharTok{\%\textgreater{}\%}
  \FunctionTok{components}\NormalTok{() }\SpecialCharTok{\%\textgreater{}\%}
  \FunctionTok{autoplot}\NormalTok{()}
\end{Highlighting}
\end{Shaded}

\includegraphics{DATA624_Project1_files/figure-latex/unnamed-chunk-17-1.pdf}

\begin{Shaded}
\begin{Highlighting}[]
\FunctionTok{ndiffs}\NormalTok{(atm1\_df}\SpecialCharTok{$}\NormalTok{ATM1)}
\end{Highlighting}
\end{Shaded}

\begin{verbatim}
## [1] 0
\end{verbatim}

\begin{Shaded}
\begin{Highlighting}[]
\NormalTok{atm1\_df }\SpecialCharTok{\%\textgreater{}\%} 
  \FunctionTok{ACF}\NormalTok{(ATM1, }\AttributeTok{lag\_max =} \DecValTok{30}\NormalTok{) }\SpecialCharTok{\%\textgreater{}\%} 
  \FunctionTok{autoplot}\NormalTok{()}
\end{Highlighting}
\end{Shaded}

\includegraphics{DATA624_Project1_files/figure-latex/unnamed-chunk-19-1.pdf}

The STL decomposition wasn't as telling as I would have liked, however
the ACF plot presents lags at 2, 5, and 7. I believe, given the week
starts on Sunday, that this represents Monday, Thursday and Saturday as
the days with the most lag. 7 has shown the value with the most
significant lag. There is a decreasing trend with the ACF plot, and
supports that the data is non-stationary would require differencing
however \(r_ 1's\) small value and the results of the \texttt{ndiff()}
function, showing the first number of differences as 0, negates that
suspicion.

\hypertarget{arima}{%
\paragraph{ARIMA}\label{arima}}

Seasonal naive method was my preferred choice considering the
seasonality, and so we can use the prior time period's withdrawals to
conduct our forecast, but I also like to default to \texttt{Auto\ ARIMA}
for the optimized selection. I assume ETS and ARIMA wont perform as well
but will await for the comparisons. Below we filter out the data
residing in May, the month we are forecasting.

\begin{Shaded}
\begin{Highlighting}[]
\CommentTok{\# train}
\NormalTok{atm1\_train }\OtherTok{\textless{}{-}}\NormalTok{ atm1\_df }\SpecialCharTok{\%\textgreater{}\%}
  \FunctionTok{filter}\NormalTok{(DATE }\SpecialCharTok{\textless{}=} \StringTok{"2010{-}04{-}01"}\NormalTok{)}


\NormalTok{atm1\_fit }\OtherTok{\textless{}{-}}\NormalTok{ atm1\_train }\SpecialCharTok{\%\textgreater{}\%}
  \FunctionTok{model}\NormalTok{(}
    \AttributeTok{SNAIVE =} \FunctionTok{SNAIVE}\NormalTok{(ATM1),}
    \AttributeTok{ETS =} \FunctionTok{ETS}\NormalTok{(ATM1),}
    \AttributeTok{ARIMA =} \FunctionTok{ARIMA}\NormalTok{(ATM1),}
    \StringTok{\textasciigrave{}}\AttributeTok{Auto ARIMA}\StringTok{\textasciigrave{}} \OtherTok{=} \FunctionTok{ARIMA}\NormalTok{(ATM1, }\AttributeTok{stepwise =} \ConstantTok{FALSE}\NormalTok{, }\AttributeTok{approx =} \ConstantTok{FALSE}\NormalTok{)}
\NormalTok{  )}

\CommentTok{\# forecast April}
\NormalTok{atm1\_forecast }\OtherTok{\textless{}{-}}\NormalTok{ atm1\_fit }\SpecialCharTok{\%\textgreater{}\%}
  \FunctionTok{forecast}\NormalTok{(}\AttributeTok{h =} \DecValTok{30}\NormalTok{)}

\CommentTok{\#plot}
\NormalTok{atm1\_forecast }\SpecialCharTok{\%\textgreater{}\%}
  \FunctionTok{autoplot}\NormalTok{(atm1\_df, }\AttributeTok{level =} \ConstantTok{NULL}\NormalTok{)}\SpecialCharTok{+}
  \FunctionTok{facet\_wrap}\NormalTok{( }\SpecialCharTok{\textasciitilde{}}\NormalTok{ .model, }\AttributeTok{scales =} \StringTok{"free\_y"}\NormalTok{) }\SpecialCharTok{+}
  \FunctionTok{guides}\NormalTok{(}\AttributeTok{colour =} \FunctionTok{guide\_legend}\NormalTok{(}\AttributeTok{title =} \StringTok{"Forecast"}\NormalTok{))}\SpecialCharTok{+}
  \FunctionTok{labs}\NormalTok{(}\AttributeTok{title=} \StringTok{"ATM1 Forecasts | April"}\NormalTok{) }\SpecialCharTok{+}
  \FunctionTok{xlab}\NormalTok{(}\StringTok{"Date"}\NormalTok{) }\SpecialCharTok{+}
  \FunctionTok{ylab}\NormalTok{(}\StringTok{"$$$ (In Hundreds)"}\NormalTok{) }
\end{Highlighting}
\end{Shaded}

\includegraphics{DATA624_Project1_files/figure-latex/unnamed-chunk-20-1.pdf}

\begin{Shaded}
\begin{Highlighting}[]
\CommentTok{\# RMSE}
\FunctionTok{accuracy}\NormalTok{(atm1\_forecast, atm1\_df) }\SpecialCharTok{\%\textgreater{}\%}
  \FunctionTok{select}\NormalTok{(.model, RMSE}\SpecialCharTok{:}\NormalTok{MAPE)}
\end{Highlighting}
\end{Shaded}

\begin{verbatim}
## # A tibble: 4 x 5
##   .model      RMSE   MAE   MPE  MAPE
##   <chr>      <dbl> <dbl> <dbl> <dbl>
## 1 ARIMA       12.6 10.0  -85.8  88.9
## 2 Auto ARIMA  13.0 10.1  -98.9 102. 
## 3 ETS         12.1  9.55 -64.5  67.8
## 4 SNAIVE      16.8 14.5  -69.5  76.6
\end{verbatim}

When interpreting the results, the model with the lowest RMSE and MAE
value and the MPE and MAPE values closes to zero the best performing.
This is true in all cases for ETS indicating it is the best performing.

\hypertarget{forecast}{%
\paragraph{Forecast}\label{forecast}}

** Reference**

\begin{verbatim}
                      aus_economy |>
                        model(ETS(Population)) |>
                        forecast(h = "5 years") |>
                        autoplot(aus_economy |> filter(Year >= 2000)) +
                        labs(title = "Australian population",
                             y = "People (millions)")
\end{verbatim}

\begin{Shaded}
\begin{Highlighting}[]
\CommentTok{\# remade the model from source}
\NormalTok{atm1\_fit\_ets }\OtherTok{\textless{}{-}}\NormalTok{ atm1\_df }\SpecialCharTok{\%\textgreater{}\%} 
  \FunctionTok{model}\NormalTok{(}\AttributeTok{ETS =} \FunctionTok{ETS}\NormalTok{(ATM1))}

\NormalTok{atm1\_forecast\_ets }\OtherTok{\textless{}{-}}\NormalTok{ atm1\_fit\_ets }\SpecialCharTok{\%\textgreater{}\%} 
  \FunctionTok{forecast}\NormalTok{(}\AttributeTok{h=}\DecValTok{30}\NormalTok{)}

\NormalTok{atm1\_forecast\_ets }\SpecialCharTok{\%\textgreater{}\%} 
  \FunctionTok{autoplot}\NormalTok{(atm1\_df) }\SpecialCharTok{+}
  \FunctionTok{labs}\NormalTok{(}\AttributeTok{title =} \StringTok{"ATM1 Forecast (ETS) | May"}\NormalTok{,}
       \AttributeTok{y =} \StringTok{"$$$ (in Hundreds)"}\NormalTok{)}
\end{Highlighting}
\end{Shaded}

\includegraphics{DATA624_Project1_files/figure-latex/unnamed-chunk-22-1.pdf}

\begin{Shaded}
\begin{Highlighting}[]
\NormalTok{(atm1\_forecast\_results }\OtherTok{\textless{}{-}} 
  \FunctionTok{as.data.frame}\NormalTok{(atm1\_forecast\_ets) }\SpecialCharTok{\%\textgreater{}\%}
    \FunctionTok{select}\NormalTok{(DATE, .mean) }\SpecialCharTok{\%\textgreater{}\%} 
      \FunctionTok{rename}\NormalTok{(}\AttributeTok{Date =}\NormalTok{ DATE, }\AttributeTok{Cash =}\NormalTok{ .mean)}\SpecialCharTok{\%\textgreater{}\%}
        \FunctionTok{mutate}\NormalTok{(}\AttributeTok{Cash=}\FunctionTok{round}\NormalTok{(Cash,}\DecValTok{2}\NormalTok{)))}
\end{Highlighting}
\end{Shaded}

\begin{verbatim}
##          Date   Cash
## 1  2010-05-01  87.05
## 2  2010-05-02 100.76
## 3  2010-05-03  73.11
## 4  2010-05-04   5.74
## 5  2010-05-05 100.13
## 6  2010-05-06  79.43
## 7  2010-05-07  85.60
## 8  2010-05-08  87.05
## 9  2010-05-09 100.76
## 10 2010-05-10  73.11
## 11 2010-05-11   5.74
## 12 2010-05-12 100.13
## 13 2010-05-13  79.43
## 14 2010-05-14  85.60
## 15 2010-05-15  87.05
## 16 2010-05-16 100.76
## 17 2010-05-17  73.11
## 18 2010-05-18   5.74
## 19 2010-05-19 100.13
## 20 2010-05-20  79.43
## 21 2010-05-21  85.60
## 22 2010-05-22  87.05
## 23 2010-05-23 100.76
## 24 2010-05-24  73.11
## 25 2010-05-25   5.74
## 26 2010-05-26 100.13
## 27 2010-05-27  79.43
## 28 2010-05-28  85.60
## 29 2010-05-29  87.05
## 30 2010-05-30 100.76
\end{verbatim}

\hypertarget{atm2}{%
\subsubsection{ATM2}\label{atm2}}

\hypertarget{stl-decomposition-1}{%
\paragraph{STL Decomposition}\label{stl-decomposition-1}}

\begin{Shaded}
\begin{Highlighting}[]
\NormalTok{atm2\_df }\OtherTok{\textless{}{-}}\NormalTok{ atm\_df }\SpecialCharTok{\%\textgreater{}\%} 
\NormalTok{  dplyr}\SpecialCharTok{::}\FunctionTok{select}\NormalTok{(DATE, ATM2)}

\NormalTok{atm2\_df }\SpecialCharTok{\%\textgreater{}\%}
  \FunctionTok{model}\NormalTok{(}
    \FunctionTok{STL}\NormalTok{(ATM2 }\SpecialCharTok{\textasciitilde{}} \FunctionTok{trend}\NormalTok{(}\AttributeTok{window =} \DecValTok{7}\NormalTok{) }\SpecialCharTok{+}
                   \FunctionTok{season}\NormalTok{(}\AttributeTok{window =} \StringTok{"periodic"}\NormalTok{),}
    \AttributeTok{robust =} \ConstantTok{TRUE}\NormalTok{)) }\SpecialCharTok{\%\textgreater{}\%}
  \FunctionTok{components}\NormalTok{() }\SpecialCharTok{\%\textgreater{}\%}
  \FunctionTok{autoplot}\NormalTok{()}
\end{Highlighting}
\end{Shaded}

\includegraphics{DATA624_Project1_files/figure-latex/unnamed-chunk-24-1.pdf}

\begin{Shaded}
\begin{Highlighting}[]
\FunctionTok{ndiffs}\NormalTok{(atm2\_df}\SpecialCharTok{$}\NormalTok{ATM2)}
\end{Highlighting}
\end{Shaded}

\begin{verbatim}
## [1] 1
\end{verbatim}

\begin{Shaded}
\begin{Highlighting}[]
\FunctionTok{unitroot\_ndiffs}\NormalTok{(atm2\_df}\SpecialCharTok{$}\NormalTok{ATM2)}
\end{Highlighting}
\end{Shaded}

\begin{verbatim}
## ndiffs 
##      1
\end{verbatim}

\begin{Shaded}
\begin{Highlighting}[]
\NormalTok{atm2\_df }\SpecialCharTok{\%\textgreater{}\%} 
  \FunctionTok{ACF}\NormalTok{(ATM2, }\AttributeTok{lag\_max =} \DecValTok{30}\NormalTok{) }\SpecialCharTok{\%\textgreater{}\%} 
  \FunctionTok{autoplot}\NormalTok{()}
\end{Highlighting}
\end{Shaded}

\includegraphics{DATA624_Project1_files/figure-latex/unnamed-chunk-27-1.pdf}

The approach with ATM2 is a rinse and repeat but in this case
differencing is needed and achieved with the below code

\begin{Shaded}
\begin{Highlighting}[]
\NormalTok{atm2\_df }\OtherTok{\textless{}{-}}\NormalTok{ atm2\_df }\SpecialCharTok{\%\textgreater{}\%} 
  \FunctionTok{mutate}\NormalTok{(}\AttributeTok{diff\_ATM2=} \FunctionTok{difference}\NormalTok{(ATM2))}
\end{Highlighting}
\end{Shaded}

\hypertarget{arima-1}{%
\paragraph{ARIMA}\label{arima-1}}

Below we again filter out data and identify our best model but include
both differenced and non-differenced data.

\begin{Shaded}
\begin{Highlighting}[]
\NormalTok{atm2\_train }\OtherTok{\textless{}{-}}\NormalTok{ atm2\_df }\SpecialCharTok{\%\textgreater{}\%}
  \FunctionTok{filter}\NormalTok{(DATE }\SpecialCharTok{\textless{}=} \StringTok{"2010{-}04{-}01"}\NormalTok{)}

\CommentTok{\#run seasonal related models without the differenced data}
\NormalTok{atm2\_fit\_nondiff }\OtherTok{\textless{}{-}}\NormalTok{ atm2\_train }\SpecialCharTok{\%\textgreater{}\%}
  \FunctionTok{model}\NormalTok{(}
    \AttributeTok{SNAIVE =} \FunctionTok{SNAIVE}\NormalTok{(ATM2),}
    \AttributeTok{ETS =} \FunctionTok{ETS}\NormalTok{(ATM2),}
\NormalTok{  )}

\CommentTok{\#run models with differenced data}
\NormalTok{atm2\_fit\_diff }\OtherTok{\textless{}{-}}\NormalTok{ atm2\_train }\SpecialCharTok{\%\textgreater{}\%}
  \FunctionTok{slice}\NormalTok{(}\DecValTok{2}\SpecialCharTok{:}\DecValTok{336}\NormalTok{) }\SpecialCharTok{\%\textgreater{}\%} 
  \FunctionTok{model}\NormalTok{(}
    \AttributeTok{ETS\_diff =} \FunctionTok{ETS}\NormalTok{(diff\_ATM2),}
    \AttributeTok{ARIMA =} \FunctionTok{ARIMA}\NormalTok{(diff\_ATM2),}
   \StringTok{\textasciigrave{}}\AttributeTok{Auto ARIMA}\StringTok{\textasciigrave{}} \OtherTok{=} \FunctionTok{ARIMA}\NormalTok{(diff\_ATM2, }\AttributeTok{stepwise =} \ConstantTok{FALSE}\NormalTok{, }\AttributeTok{approx =} \ConstantTok{FALSE}\NormalTok{)}
\NormalTok{  )}

\CommentTok{\#forecast\_ATM2 April}
\NormalTok{atm2\_forecast\_nondiff }\OtherTok{\textless{}{-}}\NormalTok{ atm2\_fit\_nondiff }\SpecialCharTok{\%\textgreater{}\%}
  \FunctionTok{forecast}\NormalTok{(}\AttributeTok{h =} \DecValTok{30}\NormalTok{)}

\CommentTok{\#forecast\_ATM2 April}
\NormalTok{atm2\_\_forecast\_diff }\OtherTok{\textless{}{-}}\NormalTok{ atm2\_fit\_diff }\SpecialCharTok{\%\textgreater{}\%}
  \FunctionTok{forecast}\NormalTok{(}\AttributeTok{h =} \DecValTok{30}\NormalTok{)}

\CommentTok{\#plot}
\NormalTok{atm2\_forecast\_nondiff }\SpecialCharTok{\%\textgreater{}\%}
  \FunctionTok{autoplot}\NormalTok{(atm2\_df, }\AttributeTok{level =} \ConstantTok{NULL}\NormalTok{)}\SpecialCharTok{+}
  \FunctionTok{facet\_wrap}\NormalTok{( }\SpecialCharTok{\textasciitilde{}}\NormalTok{ .model, }\AttributeTok{scales =} \StringTok{"free\_y"}\NormalTok{) }\SpecialCharTok{+}
  \FunctionTok{guides}\NormalTok{(}\AttributeTok{colour =} \FunctionTok{guide\_legend}\NormalTok{(}\AttributeTok{title =} \StringTok{"Forecast"}\NormalTok{))}\SpecialCharTok{+}
  \FunctionTok{labs}\NormalTok{(}\AttributeTok{title=} \StringTok{"ATM2 Forecasts | April"}\NormalTok{) }\SpecialCharTok{+}
  \FunctionTok{xlab}\NormalTok{(}\StringTok{"Date"}\NormalTok{) }\SpecialCharTok{+}
  \FunctionTok{ylab}\NormalTok{(}\StringTok{"$$$ (In Hundreds)"}\NormalTok{) }
\end{Highlighting}
\end{Shaded}

\includegraphics{DATA624_Project1_files/figure-latex/unnamed-chunk-29-1.pdf}

\begin{Shaded}
\begin{Highlighting}[]
\CommentTok{\#plot 2}
\NormalTok{atm2\_\_forecast\_diff }\SpecialCharTok{\%\textgreater{}\%}
  \FunctionTok{autoplot}\NormalTok{(atm2\_df, }\AttributeTok{level =} \ConstantTok{NULL}\NormalTok{)}\SpecialCharTok{+}
  \FunctionTok{facet\_wrap}\NormalTok{( }\SpecialCharTok{\textasciitilde{}}\NormalTok{ .model, }\AttributeTok{scales =} \StringTok{"free\_y"}\NormalTok{) }\SpecialCharTok{+}
  \FunctionTok{guides}\NormalTok{(}\AttributeTok{colour =} \FunctionTok{guide\_legend}\NormalTok{(}\AttributeTok{title =} \StringTok{"Forecast"}\NormalTok{))}\SpecialCharTok{+}
  \FunctionTok{labs}\NormalTok{(}\AttributeTok{title=} \StringTok{"ATM2 Forecasts | April"}\NormalTok{) }\SpecialCharTok{+}
  \FunctionTok{xlab}\NormalTok{(}\StringTok{"Date"}\NormalTok{) }\SpecialCharTok{+}
  \FunctionTok{ylab}\NormalTok{(}\StringTok{"Cash"}\NormalTok{)}
\end{Highlighting}
\end{Shaded}

\includegraphics{DATA624_Project1_files/figure-latex/unnamed-chunk-29-2.pdf}

\begin{Shaded}
\begin{Highlighting}[]
\FunctionTok{accuracy}\NormalTok{(atm2\_forecast\_nondiff, atm2\_df) }\SpecialCharTok{\%\textgreater{}\%}
  \FunctionTok{select}\NormalTok{(.model, RMSE}\SpecialCharTok{:}\NormalTok{MAPE)}
\end{Highlighting}
\end{Shaded}

\begin{verbatim}
## # A tibble: 2 x 5
##   .model  RMSE   MAE   MPE  MAPE
##   <chr>  <dbl> <dbl> <dbl> <dbl>
## 1 ETS     19.0  13.7 -29.3  59.4
## 2 SNAIVE  26.0  16.9  32.3  45.6
\end{verbatim}

\begin{Shaded}
\begin{Highlighting}[]
\FunctionTok{accuracy}\NormalTok{(atm2\_\_forecast\_diff, atm2\_df) }\SpecialCharTok{\%\textgreater{}\%}
  \FunctionTok{select}\NormalTok{(.model, RMSE}\SpecialCharTok{:}\NormalTok{MAPE)}
\end{Highlighting}
\end{Shaded}

\begin{verbatim}
## # A tibble: 3 x 5
##   .model      RMSE   MAE   MPE  MAPE
##   <chr>      <dbl> <dbl> <dbl> <dbl>
## 1 ARIMA       26.2  19.5  228.  239.
## 2 Auto ARIMA  25.2  19.1  234.  242.
## 3 ETS_diff    25.0  19.1  220.  229.
\end{verbatim}

Among the reuslts, the non-difference ETS model had the lowest RMSE \&
MAE, and MPE \& MAPE closest to zero, making it the optimal choice.

\hypertarget{forecast-1}{%
\paragraph{Forecast}\label{forecast-1}}

\begin{Shaded}
\begin{Highlighting}[]
\NormalTok{atm2\_fit\_ets }\OtherTok{\textless{}{-}}\NormalTok{ atm2\_df }\SpecialCharTok{\%\textgreater{}\%} 
  \FunctionTok{model}\NormalTok{(}
    \AttributeTok{ETS =} \FunctionTok{ETS}\NormalTok{(ATM2))}

\CommentTok{\#generate the values}
\NormalTok{atm2\_forecast\_ets }\OtherTok{\textless{}{-}}\NormalTok{ atm2\_fit\_ets }\SpecialCharTok{\%\textgreater{}\%} 
  \FunctionTok{forecast}\NormalTok{(}\AttributeTok{h=}\DecValTok{30}\NormalTok{)}

\CommentTok{\#plot}
\NormalTok{atm2\_forecast\_ets }\SpecialCharTok{\%\textgreater{}\%} 
  \FunctionTok{autoplot}\NormalTok{(atm2\_df) }\SpecialCharTok{+}
  \FunctionTok{labs}\NormalTok{(}\AttributeTok{title =} \StringTok{"ATM2 {-} ETS Forecast | May 2010"}\NormalTok{,}
       \AttributeTok{y =} \StringTok{"$$$ (In Hundreds)"}\NormalTok{)}
\end{Highlighting}
\end{Shaded}

\includegraphics{DATA624_Project1_files/figure-latex/unnamed-chunk-31-1.pdf}

\begin{Shaded}
\begin{Highlighting}[]
\NormalTok{(atm2\_forecast\_results }\OtherTok{\textless{}{-}} 
  \FunctionTok{as.data.frame}\NormalTok{(atm2\_forecast\_ets) }\SpecialCharTok{\%\textgreater{}\%}
    \FunctionTok{select}\NormalTok{(DATE, .mean) }\SpecialCharTok{\%\textgreater{}\%} 
      \FunctionTok{rename}\NormalTok{(}\AttributeTok{Date =}\NormalTok{ DATE, }\AttributeTok{Cash =}\NormalTok{ .mean)}\SpecialCharTok{\%\textgreater{}\%}
        \FunctionTok{mutate}\NormalTok{(}\AttributeTok{Cash=}\FunctionTok{round}\NormalTok{(Cash,}\DecValTok{2}\NormalTok{)))}
\end{Highlighting}
\end{Shaded}

\begin{verbatim}
##          Date   Cash
## 1  2010-05-01  68.35
## 2  2010-05-02  74.19
## 3  2010-05-03  11.09
## 4  2010-05-04   2.14
## 5  2010-05-05 101.60
## 6  2010-05-06  92.38
## 7  2010-05-07  68.98
## 8  2010-05-08  68.35
## 9  2010-05-09  74.19
## 10 2010-05-10  11.09
## 11 2010-05-11   2.14
## 12 2010-05-12 101.60
## 13 2010-05-13  92.38
## 14 2010-05-14  68.98
## 15 2010-05-15  68.35
## 16 2010-05-16  74.19
## 17 2010-05-17  11.09
## 18 2010-05-18   2.14
## 19 2010-05-19 101.60
## 20 2010-05-20  92.38
## 21 2010-05-21  68.98
## 22 2010-05-22  68.35
## 23 2010-05-23  74.19
## 24 2010-05-24  11.09
## 25 2010-05-25   2.14
## 26 2010-05-26 101.60
## 27 2010-05-27  92.38
## 28 2010-05-28  68.98
## 29 2010-05-29  68.35
## 30 2010-05-30  74.19
\end{verbatim}

\hypertarget{atm3}{%
\subsubsection{ATM3}\label{atm3}}

ATM3 was ultimately omitted, considering the limited date range and
skewed distributions. It can be considered when more data is provided.

\hypertarget{atm4}{%
\subsubsection{ATM4}\label{atm4}}

\hypertarget{stl-decomposition-2}{%
\paragraph{STL Decomposition}\label{stl-decomposition-2}}

\begin{Shaded}
\begin{Highlighting}[]
\NormalTok{atm4\_df }\OtherTok{\textless{}{-}}\NormalTok{ atm\_df }\SpecialCharTok{\%\textgreater{}\%} 
  \FunctionTok{select}\NormalTok{(DATE, ATM4)}

\NormalTok{atm4\_df }\SpecialCharTok{\%\textgreater{}\%}
  \FunctionTok{model}\NormalTok{(}
    \FunctionTok{STL}\NormalTok{(ATM4 }\SpecialCharTok{\textasciitilde{}} \FunctionTok{trend}\NormalTok{(}\AttributeTok{window =} \DecValTok{7}\NormalTok{) }\SpecialCharTok{+}
                   \FunctionTok{season}\NormalTok{(}\AttributeTok{window =} \StringTok{"periodic"}\NormalTok{),}
    \AttributeTok{robust =} \ConstantTok{TRUE}\NormalTok{)) }\SpecialCharTok{\%\textgreater{}\%}
  \FunctionTok{components}\NormalTok{() }\SpecialCharTok{\%\textgreater{}\%}
  \FunctionTok{autoplot}\NormalTok{()}
\end{Highlighting}
\end{Shaded}

\includegraphics{DATA624_Project1_files/figure-latex/unnamed-chunk-33-1.pdf}

Considering the variance from the time series, I decided to tranform the
data before forecasting using box-cox transformation

\hypertarget{box-cox}{%
\paragraph{Box-Cox}\label{box-cox}}

\textbf{Reference}

\href{https://otexts.com/fpp3/transformations.html}{Forecasting
Principles and Practice}

\begin{verbatim}
      lambda <- aus_production |>
        features(Gas, features = guerrero) |>
        pull(lambda_guerrero)
      aus_production |>
        autoplot(box_cox(Gas, lambda)) +
        labs(y = "",
             title = latex2exp::TeX(paste0(
               "Transformed gas production with $\\lambda$ = ",
               round(lambda,2))))
\end{verbatim}

\begin{Shaded}
\begin{Highlighting}[]
\NormalTok{(atm4\_lambda }\OtherTok{\textless{}{-}}\NormalTok{ atm4\_df }\SpecialCharTok{\%\textgreater{}\%}
  \FunctionTok{features}\NormalTok{(ATM4, }\AttributeTok{features =}\NormalTok{ guerrero) }\SpecialCharTok{\%\textgreater{}\%}
  \FunctionTok{pull}\NormalTok{(lambda\_guerrero))}
\end{Highlighting}
\end{Shaded}

\begin{verbatim}
## [1] -0.0737252
\end{verbatim}

\begin{Shaded}
\begin{Highlighting}[]
\NormalTok{atm4\_transformed }\OtherTok{\textless{}{-}} \FunctionTok{BoxCox}\NormalTok{(atm4\_df}\SpecialCharTok{$}\NormalTok{ATM4, }\AttributeTok{lambda =}\NormalTok{ atm4\_lambda)}

\CommentTok{\# Extract the transformed data}

\NormalTok{atm4\_df}\SpecialCharTok{$}\NormalTok{ATM4\_T}\OtherTok{\textless{}{-}}\NormalTok{atm4\_transformed}

\CommentTok{\#plot}
\NormalTok{atm4\_df}\SpecialCharTok{\%\textgreater{}\%} 
  \FunctionTok{autoplot}\NormalTok{(ATM4\_T) }
\end{Highlighting}
\end{Shaded}

\includegraphics{DATA624_Project1_files/figure-latex/unnamed-chunk-35-1.pdf}

\begin{Shaded}
\begin{Highlighting}[]
\FunctionTok{ndiffs}\NormalTok{(atm4\_df}\SpecialCharTok{$}\NormalTok{ATM4)}
\end{Highlighting}
\end{Shaded}

\begin{verbatim}
## [1] 0
\end{verbatim}

\begin{Shaded}
\begin{Highlighting}[]
\FunctionTok{ndiffs}\NormalTok{(atm4\_df}\SpecialCharTok{$}\NormalTok{ATM4\_T)}
\end{Highlighting}
\end{Shaded}

\begin{verbatim}
## [1] 0
\end{verbatim}

\begin{Shaded}
\begin{Highlighting}[]
\NormalTok{atm4\_df }\SpecialCharTok{\%\textgreater{}\%} 
  \FunctionTok{ACF}\NormalTok{(ATM4\_T, }\AttributeTok{lag\_max =} \DecValTok{28}\NormalTok{) }\SpecialCharTok{\%\textgreater{}\%} 
  \FunctionTok{autoplot}\NormalTok{()}
\end{Highlighting}
\end{Shaded}

\includegraphics{DATA624_Project1_files/figure-latex/unnamed-chunk-37-1.pdf}

Using ndiff() we identify that theres no need for differencing, and the
ACF shows

The ACF plot below suggest lags 7 consistently and on 2 other occasions
in different periods. Despite the ndiff() function resulting in 0, if
believe this does require differencing using the transformed data.

\begin{Shaded}
\begin{Highlighting}[]
\NormalTok{atm4\_df }\OtherTok{\textless{}{-}}\NormalTok{ atm4\_df }\SpecialCharTok{\%\textgreater{}\%} 
  \FunctionTok{mutate}\NormalTok{(}\AttributeTok{diff\_ATM4=} \FunctionTok{difference}\NormalTok{(ATM4\_T))}
\end{Highlighting}
\end{Shaded}

\hypertarget{arima-2}{%
\paragraph{ARIMA}\label{arima-2}}

\begin{Shaded}
\begin{Highlighting}[]
\NormalTok{atm4\_train }\OtherTok{\textless{}{-}}\NormalTok{ atm4\_df }\SpecialCharTok{\%\textgreater{}\%}
  \FunctionTok{filter}\NormalTok{(DATE }\SpecialCharTok{\textless{}=} \StringTok{"2010{-}04{-}01"}\NormalTok{)}

\CommentTok{\#run seasonal related models without the differenced data}
\NormalTok{atm4\_fit\_nondiff }\OtherTok{\textless{}{-}}\NormalTok{ atm4\_train }\SpecialCharTok{\%\textgreater{}\%}
  \FunctionTok{model}\NormalTok{(}
    \AttributeTok{SNAIVE =} \FunctionTok{SNAIVE}\NormalTok{(ATM4\_T),}
    \AttributeTok{ETS =} \FunctionTok{ETS}\NormalTok{(ATM4\_T),}
\NormalTok{  )}

\CommentTok{\#run models with differenced data}
\NormalTok{atm4\_fit\_diff }\OtherTok{\textless{}{-}}\NormalTok{ atm4\_train }\SpecialCharTok{\%\textgreater{}\%}
  \FunctionTok{slice}\NormalTok{(}\DecValTok{2}\SpecialCharTok{:}\DecValTok{336}\NormalTok{) }\SpecialCharTok{\%\textgreater{}\%} 
  \FunctionTok{model}\NormalTok{(}
    \AttributeTok{ETS\_diff =} \FunctionTok{ETS}\NormalTok{(diff\_ATM4),}
    \AttributeTok{ARIMA =} \FunctionTok{ARIMA}\NormalTok{(diff\_ATM4),}
   \StringTok{\textasciigrave{}}\AttributeTok{Auto ARIMA}\StringTok{\textasciigrave{}} \OtherTok{=} \FunctionTok{ARIMA}\NormalTok{(diff\_ATM4, }\AttributeTok{stepwise =} \ConstantTok{FALSE}\NormalTok{, }\AttributeTok{approx =} \ConstantTok{FALSE}\NormalTok{)}
\NormalTok{  )}

\CommentTok{\#forecast\_ATM2 April}
\NormalTok{atm4\_forecast\_nondiff }\OtherTok{\textless{}{-}}\NormalTok{ atm4\_fit\_nondiff }\SpecialCharTok{\%\textgreater{}\%}
  \FunctionTok{forecast}\NormalTok{(}\AttributeTok{h =} \DecValTok{30}\NormalTok{)}

\CommentTok{\#forecast\_ATM2 April}
\NormalTok{atm4\_\_forecast\_diff }\OtherTok{\textless{}{-}}\NormalTok{ atm4\_fit\_diff }\SpecialCharTok{\%\textgreater{}\%}
  \FunctionTok{forecast}\NormalTok{(}\AttributeTok{h =} \DecValTok{30}\NormalTok{)}

\CommentTok{\#plot}
\NormalTok{atm4\_forecast\_nondiff }\SpecialCharTok{\%\textgreater{}\%}
  \FunctionTok{autoplot}\NormalTok{(atm4\_df, }\AttributeTok{level =} \ConstantTok{NULL}\NormalTok{)}\SpecialCharTok{+}
  \FunctionTok{facet\_wrap}\NormalTok{( }\SpecialCharTok{\textasciitilde{}}\NormalTok{ .model, }\AttributeTok{scales =} \StringTok{"free\_y"}\NormalTok{) }\SpecialCharTok{+}
  \FunctionTok{guides}\NormalTok{(}\AttributeTok{colour =} \FunctionTok{guide\_legend}\NormalTok{(}\AttributeTok{title =} \StringTok{"Forecast"}\NormalTok{))}\SpecialCharTok{+}
  \FunctionTok{labs}\NormalTok{(}\AttributeTok{title=} \StringTok{"ATM4 Forecasts | April"}\NormalTok{) }\SpecialCharTok{+}
  \FunctionTok{xlab}\NormalTok{(}\StringTok{"Date"}\NormalTok{) }\SpecialCharTok{+}
  \FunctionTok{ylab}\NormalTok{(}\StringTok{"$$$ (In Hundreds)"}\NormalTok{) }
\end{Highlighting}
\end{Shaded}

\includegraphics{DATA624_Project1_files/figure-latex/unnamed-chunk-39-1.pdf}

\begin{Shaded}
\begin{Highlighting}[]
\CommentTok{\#plot 2}
\NormalTok{atm4\_\_forecast\_diff }\SpecialCharTok{\%\textgreater{}\%}
  \FunctionTok{autoplot}\NormalTok{(atm4\_df, }\AttributeTok{level =} \ConstantTok{NULL}\NormalTok{)}\SpecialCharTok{+}
  \FunctionTok{facet\_wrap}\NormalTok{( }\SpecialCharTok{\textasciitilde{}}\NormalTok{ .model, }\AttributeTok{scales =} \StringTok{"free\_y"}\NormalTok{) }\SpecialCharTok{+}
  \FunctionTok{guides}\NormalTok{(}\AttributeTok{colour =} \FunctionTok{guide\_legend}\NormalTok{(}\AttributeTok{title =} \StringTok{"Forecast"}\NormalTok{))}\SpecialCharTok{+}
  \FunctionTok{labs}\NormalTok{(}\AttributeTok{title=} \StringTok{"ATM4 Forecasts | April"}\NormalTok{) }\SpecialCharTok{+}
  \FunctionTok{xlab}\NormalTok{(}\StringTok{"Date"}\NormalTok{) }\SpecialCharTok{+}
  \FunctionTok{ylab}\NormalTok{(}\StringTok{"Cash"}\NormalTok{)}
\end{Highlighting}
\end{Shaded}

\includegraphics{DATA624_Project1_files/figure-latex/unnamed-chunk-39-2.pdf}

\begin{Shaded}
\begin{Highlighting}[]
\FunctionTok{accuracy}\NormalTok{(atm4\_forecast\_nondiff, atm4\_df) }\SpecialCharTok{\%\textgreater{}\%}
  \FunctionTok{select}\NormalTok{(.model, RMSE}\SpecialCharTok{:}\NormalTok{MAPE)}
\end{Highlighting}
\end{Shaded}

\begin{verbatim}
## # A tibble: 2 x 5
##   .model  RMSE   MAE   MPE  MAPE
##   <chr>  <dbl> <dbl> <dbl> <dbl>
## 1 ETS    1.07  0.740 -22.3  32.8
## 2 SNAIVE 0.861 0.510 -19.5  22.7
\end{verbatim}

\begin{Shaded}
\begin{Highlighting}[]
\FunctionTok{accuracy}\NormalTok{(atm4\_\_forecast\_diff, atm4\_df) }\SpecialCharTok{\%\textgreater{}\%}
  \FunctionTok{select}\NormalTok{(.model, RMSE}\SpecialCharTok{:}\NormalTok{MAPE)}
\end{Highlighting}
\end{Shaded}

\begin{verbatim}
## # A tibble: 3 x 5
##   .model      RMSE   MAE   MPE  MAPE
##   <chr>      <dbl> <dbl> <dbl> <dbl>
## 1 ARIMA       1.64  1.26 104.   104.
## 2 Auto ARIMA  1.66  1.26 103.   103.
## 3 ETS_diff    1.73  1.33  21.7  184.
\end{verbatim}

Of the models, SNAIVE for non differenced data was the most accurate so
I will proceed with this.

\hypertarget{forecast-2}{%
\paragraph{Forecast}\label{forecast-2}}

\begin{Shaded}
\begin{Highlighting}[]
\NormalTok{atm4\_fit\_snaive }\OtherTok{\textless{}{-}}\NormalTok{ atm4\_df }\SpecialCharTok{\%\textgreater{}\%} 
  \FunctionTok{model}\NormalTok{(}
    \AttributeTok{SNAIVE =} \FunctionTok{SNAIVE}\NormalTok{(ATM4\_T))}

\CommentTok{\#generate the values}
\NormalTok{atm4\_forecast\_snaive }\OtherTok{\textless{}{-}}\NormalTok{ atm4\_fit\_snaive }\SpecialCharTok{\%\textgreater{}\%} 
  \FunctionTok{forecast}\NormalTok{(}\AttributeTok{h=}\DecValTok{30}\NormalTok{)}

\CommentTok{\#plot}
\NormalTok{atm4\_forecast\_snaive }\SpecialCharTok{\%\textgreater{}\%} 
  \FunctionTok{autoplot}\NormalTok{(atm4\_df) }\SpecialCharTok{+}
  \FunctionTok{labs}\NormalTok{(}\AttributeTok{title =} \StringTok{"ATM2 {-} SNAIVE Forecast | May 2010"}\NormalTok{,}
       \AttributeTok{y =} \StringTok{"$$$ (In Hundreds)"}\NormalTok{)}
\end{Highlighting}
\end{Shaded}

\includegraphics{DATA624_Project1_files/figure-latex/unnamed-chunk-41-1.pdf}

\begin{Shaded}
\begin{Highlighting}[]
\NormalTok{(atm4\_forecast\_results }\OtherTok{\textless{}{-}} 
  \FunctionTok{as.data.frame}\NormalTok{(atm4\_forecast\_snaive) }\SpecialCharTok{\%\textgreater{}\%}
    \FunctionTok{select}\NormalTok{(DATE, .mean) }\SpecialCharTok{\%\textgreater{}\%} 
      \FunctionTok{rename}\NormalTok{(}\AttributeTok{Date =}\NormalTok{ DATE, }\AttributeTok{Cash =}\NormalTok{ .mean)}\SpecialCharTok{\%\textgreater{}\%}
        \FunctionTok{mutate}\NormalTok{(}\AttributeTok{Cash=}\FunctionTok{round}\NormalTok{(Cash,}\DecValTok{2}\NormalTok{)))}
\end{Highlighting}
\end{Shaded}

\begin{verbatim}
##          Date Cash
## 1  2010-05-01 1.59
## 2  2010-05-02 5.04
## 3  2010-05-03 4.85
## 4  2010-05-04 2.38
## 5  2010-05-05 4.75
## 6  2010-05-06 3.31
## 7  2010-05-07 4.96
## 8  2010-05-08 1.59
## 9  2010-05-09 5.04
## 10 2010-05-10 4.85
## 11 2010-05-11 2.38
## 12 2010-05-12 4.75
## 13 2010-05-13 3.31
## 14 2010-05-14 4.96
## 15 2010-05-15 1.59
## 16 2010-05-16 5.04
## 17 2010-05-17 4.85
## 18 2010-05-18 2.38
## 19 2010-05-19 4.75
## 20 2010-05-20 3.31
## 21 2010-05-21 4.96
## 22 2010-05-22 1.59
## 23 2010-05-23 5.04
## 24 2010-05-24 4.85
## 25 2010-05-25 2.38
## 26 2010-05-26 4.75
## 27 2010-05-27 3.31
## 28 2010-05-28 4.96
## 29 2010-05-29 1.59
## 30 2010-05-30 5.04
\end{verbatim}

\hypertarget{part-b-1}{%
\section{Part B}\label{part-b-1}}

\hypertarget{eda-cleanup-1}{%
\subsection{EDA \& Cleanup}\label{eda-cleanup-1}}

\begin{Shaded}
\begin{Highlighting}[]
\FunctionTok{str}\NormalTok{(power\_raw)}
\end{Highlighting}
\end{Shaded}

\begin{verbatim}
## tibble [192 x 3] (S3: tbl_df/tbl/data.frame)
##  $ CaseSequence: num [1:192] 733 734 735 736 737 738 739 740 741 742 ...
##  $ YYYY-MMM    : chr [1:192] "1998-Jan" "1998-Feb" "1998-Mar" "1998-Apr" ...
##  $ KWH         : num [1:192] 6862583 5838198 5420658 5010364 4665377 ...
\end{verbatim}

\begin{Shaded}
\begin{Highlighting}[]
\FunctionTok{describe}\NormalTok{(power\_raw)}
\end{Highlighting}
\end{Shaded}

\begin{verbatim}
##              vars   n      mean         sd    median   trimmed        mad
## CaseSequence    1 192     828.5      55.57     828.5     828.5      71.16
## YYYY-MMM*       2 192      96.5      55.57      96.5      96.5      71.16
## KWH             3 191 6502474.6 1447570.89 6283324.0 6439474.9 1543073.77
##                 min      max   range skew kurtosis        se
## CaseSequence    733      924     191 0.00    -1.22      4.01
## YYYY-MMM*         1      192     191 0.00    -1.22      4.01
## KWH          770523 10655730 9885207 0.17     0.45 104742.55
\end{verbatim}

\begin{Shaded}
\begin{Highlighting}[]
\FunctionTok{data.frame}\NormalTok{(power\_raw}\SpecialCharTok{$}\StringTok{\textasciigrave{}}\AttributeTok{YYYY{-}MMM}\StringTok{\textasciigrave{}}\NormalTok{[power\_raw}\SpecialCharTok{$}\NormalTok{KWH }\SpecialCharTok{\%in\%} \ConstantTok{NA}\NormalTok{])}
\end{Highlighting}
\end{Shaded}

\begin{verbatim}
##   power_raw..YYYY.MMM..power_raw.KWH..in..NA.
## 1                                    2008-Sep
\end{verbatim}

I renamed the \texttt{YYYY-MMM} for preference to \texttt{DATE}. The
cleanup is a change of type for \texttt{DATE}, removal of
\texttt{CaseSequence} as it does not help our model, and reducing our
model to values in the thousands for ease of analysis. Like before we'll
also be indexing by DATE

\begin{Shaded}
\begin{Highlighting}[]
\CommentTok{\#change variable type }
\NormalTok{power\_df }\OtherTok{\textless{}{-}}\NormalTok{ power\_raw }\SpecialCharTok{\%\textgreater{}\%} 
  \FunctionTok{mutate}\NormalTok{(}\AttributeTok{DATE =} \FunctionTok{yearmonth}\NormalTok{(}\StringTok{\textasciigrave{}}\AttributeTok{YYYY{-}MMM}\StringTok{\textasciigrave{}}\NormalTok{), }\AttributeTok{KWH =}\NormalTok{ KWH}\SpecialCharTok{/}\DecValTok{1000}\NormalTok{) }\SpecialCharTok{\%\textgreater{}\%}
  \FunctionTok{select}\NormalTok{(}\SpecialCharTok{{-}}\NormalTok{CaseSequence, }\SpecialCharTok{{-}}\StringTok{\textquotesingle{}YYYY{-}MMM\textquotesingle{}}\NormalTok{) }\SpecialCharTok{\%\textgreater{}\%} 
  \FunctionTok{tsibble}\NormalTok{(}\AttributeTok{index=}\NormalTok{ DATE)}
\end{Highlighting}
\end{Shaded}

\begin{Shaded}
\begin{Highlighting}[]
\FunctionTok{head}\NormalTok{(power\_df)}
\end{Highlighting}
\end{Shaded}

\begin{verbatim}
## # A tsibble: 6 x 2 [1M]
##     KWH     DATE
##   <dbl>    <mth>
## 1 6863. 1998 Jan
## 2 5838. 1998 Feb
## 3 5421. 1998 Mar
## 4 5010. 1998 Apr
## 5 4665. 1998 May
## 6 6467. 1998 Jun
\end{verbatim}

\begin{Shaded}
\begin{Highlighting}[]
\FunctionTok{ggplot}\NormalTok{(power\_df, }\FunctionTok{aes}\NormalTok{(}\AttributeTok{x=}\NormalTok{KWH))}\SpecialCharTok{+}
  \FunctionTok{geom\_histogram}\NormalTok{(}\AttributeTok{bins=}\DecValTok{40}\NormalTok{)}\SpecialCharTok{+}
  \FunctionTok{labs}\NormalTok{(}\AttributeTok{title =} \StringTok{"Monthly Distributions Residential Power Usage | Jan \textquotesingle{}98 {-} Dec \textquotesingle{}13"}\NormalTok{)}
\end{Highlighting}
\end{Shaded}

\includegraphics{DATA624_Project1_files/figure-latex/unnamed-chunk-48-1.pdf}

\begin{Shaded}
\begin{Highlighting}[]
\NormalTok{power\_df }\SpecialCharTok{\%\textgreater{}\%}
  \FunctionTok{autoplot}\NormalTok{(KWH) }\SpecialCharTok{+}
  \FunctionTok{labs}\NormalTok{(}\AttributeTok{title =} \StringTok{"Monthly Distributions Residential Power Usage | Jan \textquotesingle{}98 {-} Dec \textquotesingle{}13"}\NormalTok{)}
\end{Highlighting}
\end{Shaded}

\includegraphics{DATA624_Project1_files/figure-latex/unnamed-chunk-48-2.pdf}

The data has an apparent outlier and is right skewed that appears in
both plots, and resides sometime after January of 2010.

\begin{Shaded}
\begin{Highlighting}[]
\CommentTok{\# made a copy of the data first}
\NormalTok{power\_df2}\OtherTok{\textless{}{-}}\NormalTok{power\_df}
\NormalTok{power\_df2}\SpecialCharTok{$}\NormalTok{KWH }\OtherTok{\textless{}{-}} \FunctionTok{na.interp}\NormalTok{(power\_df2}\SpecialCharTok{$}\NormalTok{KWH)}
\NormalTok{power\_df2}\SpecialCharTok{$}\NormalTok{KWH }\OtherTok{\textless{}{-}} \FunctionTok{replace}\NormalTok{(power\_df2}\SpecialCharTok{$}\NormalTok{KWH, power\_df2}\SpecialCharTok{$}\NormalTok{KWH }\SpecialCharTok{==} \FunctionTok{min}\NormalTok{(power\_df2}\SpecialCharTok{$}\NormalTok{KWH),}
                          \FunctionTok{median}\NormalTok{(power\_df2}\SpecialCharTok{$}\NormalTok{KWH))}
\end{Highlighting}
\end{Shaded}

Considering the distribution, I again thought it best to replace the
missing value with the median, but considering I will be using that
method to address the outlier, I decided to use \texttt{na.interp} since
its a tool used by the author of our textbook Rob J Hydman's
\href{https://github.com/robjhyndman/forecast/blob/master/man/na.interp.Rd}{github
repo}. Regardless, the transformation below shows its still right skewed
but shows seasonality with an upward trend.

\begin{Shaded}
\begin{Highlighting}[]
\FunctionTok{ggplot}\NormalTok{(power\_df2, }\FunctionTok{aes}\NormalTok{(}\AttributeTok{x=}\NormalTok{KWH))}\SpecialCharTok{+}
  \FunctionTok{geom\_histogram}\NormalTok{()}\SpecialCharTok{+}
  \FunctionTok{labs}\NormalTok{(}\AttributeTok{title =} \StringTok{"Monthly Distributions Residential Power Usage | Jan \textquotesingle{}98 {-} Dec \textquotesingle{}13"}\NormalTok{)}
\end{Highlighting}
\end{Shaded}

\includegraphics{DATA624_Project1_files/figure-latex/unnamed-chunk-50-1.pdf}

\begin{Shaded}
\begin{Highlighting}[]
\CommentTok{\#summary}
\FunctionTok{summary}\NormalTok{(power\_df2}\SpecialCharTok{$}\NormalTok{KWH)}
\end{Highlighting}
\end{Shaded}

\begin{verbatim}
##    Min. 1st Qu.  Median    Mean 3rd Qu.    Max. 
##    4313    5444    6330    6532    7609   10656
\end{verbatim}

\begin{Shaded}
\begin{Highlighting}[]
\CommentTok{\#ts plot}
\NormalTok{power\_df2 }\SpecialCharTok{\%\textgreater{}\%}
  \FunctionTok{autoplot}\NormalTok{(KWH) }\SpecialCharTok{+}
  \FunctionTok{labs}\NormalTok{(}\AttributeTok{title =} \StringTok{"Monthly Distributions Residential Power Usage | Jan \textquotesingle{}98 {-} Dec \textquotesingle{}13"}\NormalTok{)}\SpecialCharTok{+}
  \FunctionTok{ylab}\NormalTok{(}\AttributeTok{label=} \StringTok{"KWH (Thousands)"}\NormalTok{)}
\end{Highlighting}
\end{Shaded}

\includegraphics{DATA624_Project1_files/figure-latex/unnamed-chunk-50-2.pdf}

Before forecasting I will transform the data using a Box-Cox
transformation.

\begin{Shaded}
\begin{Highlighting}[]
\CommentTok{\#get lambda}
\NormalTok{(power\_lambda }\OtherTok{\textless{}{-}}\NormalTok{ power\_df2 }\SpecialCharTok{\%\textgreater{}\%}
  \FunctionTok{features}\NormalTok{(KWH, }\AttributeTok{features =}\NormalTok{ guerrero) }\SpecialCharTok{\%\textgreater{}\%}
  \FunctionTok{pull}\NormalTok{(lambda\_guerrero))}
\end{Highlighting}
\end{Shaded}

\begin{verbatim}
## [1] -0.2130548
\end{verbatim}

\begin{Shaded}
\begin{Highlighting}[]
\NormalTok{power\_df2 }\OtherTok{\textless{}{-}}\NormalTok{ power\_df2 }\SpecialCharTok{\%\textgreater{}\%} 
    \FunctionTok{mutate}\NormalTok{(}\AttributeTok{KWH\_bc =} \FunctionTok{box\_cox}\NormalTok{(KWH, power\_lambda))}


\FunctionTok{summary}\NormalTok{(power\_df2}\SpecialCharTok{$}\NormalTok{KWH\_bc)}
\end{Highlighting}
\end{Shaded}

\begin{verbatim}
##    Min. 1st Qu.  Median    Mean 3rd Qu.    Max. 
##   3.905   3.943   3.967   3.967   3.994   4.043
\end{verbatim}

\begin{Shaded}
\begin{Highlighting}[]
\FunctionTok{ggplot}\NormalTok{(power\_df2, }\FunctionTok{aes}\NormalTok{(}\AttributeTok{x=}\NormalTok{KWH\_bc))}\SpecialCharTok{+}
  \FunctionTok{geom\_histogram}\NormalTok{()}\SpecialCharTok{+}
  \FunctionTok{labs}\NormalTok{(}\AttributeTok{title =} \StringTok{"Monthly Distributions Residential Power Usage | Jan \textquotesingle{}98 {-} Dec \textquotesingle{}13"}\NormalTok{)}
\end{Highlighting}
\end{Shaded}

\begin{verbatim}
## Don't know how to automatically pick scale for object of type <ts>. Defaulting
## to continuous.
## `stat_bin()` using `bins = 30`. Pick better value with `binwidth`.
\end{verbatim}

\includegraphics{DATA624_Project1_files/figure-latex/unnamed-chunk-53-1.pdf}

\begin{Shaded}
\begin{Highlighting}[]
\NormalTok{power\_df2 }\SpecialCharTok{\%\textgreater{}\%} 
  \FunctionTok{autoplot}\NormalTok{(KWH\_bc) }\SpecialCharTok{+}
  \FunctionTok{labs}\NormalTok{(}\AttributeTok{y =} \StringTok{"KWH in Thousands"}\NormalTok{,}
       \AttributeTok{title =} \StringTok{"Transformed KWH with Lambda = {-}0.2130548"}\NormalTok{)}
\end{Highlighting}
\end{Shaded}

\begin{verbatim}
## Don't know how to automatically pick scale for object of type <ts>. Defaulting
## to continuous.
\end{verbatim}

\includegraphics{DATA624_Project1_files/figure-latex/unnamed-chunk-53-2.pdf}

\hypertarget{stl-decomposition-3}{%
\subsection{STL Decomposition}\label{stl-decomposition-3}}

\begin{itemize}
\tightlist
\item
  STL decomposition again used to identify seasonality, variance, etc.
\item
  ndiff() and ACF will identify if differencing is needed.
\end{itemize}

\begin{Shaded}
\begin{Highlighting}[]
\NormalTok{power\_df2 }\SpecialCharTok{\%\textgreater{}\%}
  \FunctionTok{model}\NormalTok{(}
    \FunctionTok{STL}\NormalTok{(KWH\_bc }\SpecialCharTok{\textasciitilde{}} \FunctionTok{trend}\NormalTok{(}\AttributeTok{window =} \DecValTok{13}\NormalTok{) }\SpecialCharTok{+}
                   \FunctionTok{season}\NormalTok{(}\AttributeTok{window =} \StringTok{"periodic"}\NormalTok{),}
                      \AttributeTok{robust =} \ConstantTok{TRUE}\NormalTok{)) }\SpecialCharTok{\%\textgreater{}\%}
                        \FunctionTok{components}\NormalTok{() }\SpecialCharTok{\%\textgreater{}\%}
                            \FunctionTok{autoplot}\NormalTok{()}
\end{Highlighting}
\end{Shaded}

\includegraphics{DATA624_Project1_files/figure-latex/unnamed-chunk-54-1.pdf}

\begin{Shaded}
\begin{Highlighting}[]
\FunctionTok{ndiffs}\NormalTok{(power\_df2}\SpecialCharTok{$}\NormalTok{KWH\_bc)}
\end{Highlighting}
\end{Shaded}

\begin{verbatim}
## [1] 1
\end{verbatim}

\begin{Shaded}
\begin{Highlighting}[]
\NormalTok{power\_df2 }\SpecialCharTok{\%\textgreater{}\%} 
  \FunctionTok{ACF}\NormalTok{(KWH\_bc, }\AttributeTok{lag\_max =} \DecValTok{36}\NormalTok{) }\SpecialCharTok{\%\textgreater{}\%} 
  \FunctionTok{autoplot}\NormalTok{()}
\end{Highlighting}
\end{Shaded}

\includegraphics{DATA624_Project1_files/figure-latex/unnamed-chunk-56-1.pdf}

Differencing is needed.

\begin{Shaded}
\begin{Highlighting}[]
\NormalTok{diff\_power }\OtherTok{\textless{}{-}}\NormalTok{ power\_df2 }\SpecialCharTok{\%\textgreater{}\%} 
  \FunctionTok{mutate}\NormalTok{(}\AttributeTok{diff\_KWH=} \FunctionTok{difference}\NormalTok{(KWH), }\AttributeTok{diff\_KWH\_bc =} \FunctionTok{difference}\NormalTok{(KWH\_bc))}
\end{Highlighting}
\end{Shaded}

\begin{itemize}
\tightlist
\item
  Differencing created
\item
  NA and some columns need removal
\end{itemize}

\begin{Shaded}
\begin{Highlighting}[]
\NormalTok{diff\_power}\OtherTok{\textless{}{-}}\NormalTok{diff\_power}\SpecialCharTok{\%\textgreater{}\%}
  \FunctionTok{select}\NormalTok{(}\SpecialCharTok{{-}}\NormalTok{KWH, }\SpecialCharTok{{-}}\NormalTok{KWH\_bc)}\SpecialCharTok{\%\textgreater{}\%}
                        \FunctionTok{slice}\NormalTok{(}\SpecialCharTok{{-}}\DecValTok{1}\NormalTok{)}

\FunctionTok{ndiffs}\NormalTok{(diff\_power}\SpecialCharTok{$}\NormalTok{diff\_KWH\_bc)}
\end{Highlighting}
\end{Shaded}

\begin{verbatim}
## [1] 0
\end{verbatim}

\hypertarget{forecast-3}{%
\subsection{Forecast}\label{forecast-3}}

\begin{Shaded}
\begin{Highlighting}[]
\CommentTok{\#Differenced data for arima}

\CommentTok{\#split}
\NormalTok{power\_train\_diff }\OtherTok{\textless{}{-}}\NormalTok{ diff\_power }\SpecialCharTok{\%\textgreater{}\%} 
  \FunctionTok{filter}\NormalTok{(}\FunctionTok{year}\NormalTok{(DATE) }\SpecialCharTok{\textless{}} \DecValTok{2013}\NormalTok{)}

\CommentTok{\#models}
\NormalTok{power\_fit\_diff }\OtherTok{\textless{}{-}}\NormalTok{ power\_train\_diff }\SpecialCharTok{\%\textgreater{}\%} 
    \FunctionTok{model}\NormalTok{(}
    \AttributeTok{ARIMA =} \FunctionTok{ARIMA}\NormalTok{(diff\_KWH),}
    \StringTok{\textasciigrave{}}\AttributeTok{Auto ARIMA}\StringTok{\textasciigrave{}} \OtherTok{=} \FunctionTok{ARIMA}\NormalTok{(diff\_KWH, }\AttributeTok{stepwise =} \ConstantTok{FALSE}\NormalTok{, }\AttributeTok{approx =} \ConstantTok{FALSE}\NormalTok{)}
\NormalTok{  )}

\CommentTok{\#forecast of 2013}
\NormalTok{power\_forecast\_diff }\OtherTok{\textless{}{-}}\NormalTok{ power\_fit\_diff }\SpecialCharTok{\%\textgreater{}\%} 
  \FunctionTok{forecast}\NormalTok{(}\AttributeTok{h =} \StringTok{"1 year"}\NormalTok{)}

\CommentTok{\#plot}
\NormalTok{power\_forecast\_diff }\SpecialCharTok{\%\textgreater{}\%}
  \FunctionTok{autoplot}\NormalTok{(diff\_power, }\AttributeTok{level =} \ConstantTok{NULL}\NormalTok{)}\SpecialCharTok{+}
  \FunctionTok{facet\_wrap}\NormalTok{( }\SpecialCharTok{\textasciitilde{}}\NormalTok{ .model, }\AttributeTok{scales =} \StringTok{"free\_y"}\NormalTok{) }\SpecialCharTok{+}
  \FunctionTok{guides}\NormalTok{(}\AttributeTok{colour =} \FunctionTok{guide\_legend}\NormalTok{(}\AttributeTok{title =} \StringTok{"Forecast"}\NormalTok{))}\SpecialCharTok{+}
  \FunctionTok{labs}\NormalTok{(}\AttributeTok{title=} \StringTok{"KWH Forecasts | Jan \textquotesingle{}13 {-} Dec \textquotesingle{}13"}\NormalTok{)}\SpecialCharTok{+}
  \FunctionTok{xlab}\NormalTok{(}\StringTok{"Month"}\NormalTok{) }\SpecialCharTok{+}
  \FunctionTok{ylab}\NormalTok{(}\StringTok{"KWH in Thousands"}\NormalTok{) }
\end{Highlighting}
\end{Shaded}

\includegraphics{DATA624_Project1_files/figure-latex/unnamed-chunk-59-1.pdf}

\begin{Shaded}
\begin{Highlighting}[]
\CommentTok{\#split}
\NormalTok{power\_train }\OtherTok{\textless{}{-}}\NormalTok{ power\_df2 }\SpecialCharTok{\%\textgreater{}\%} 
  \FunctionTok{filter}\NormalTok{(}\FunctionTok{year}\NormalTok{(DATE) }\SpecialCharTok{\textless{}} \DecValTok{2013}\NormalTok{)}

\CommentTok{\#models}
\NormalTok{power\_fit }\OtherTok{\textless{}{-}}\NormalTok{ power\_train }\SpecialCharTok{\%\textgreater{}\%} 
    \FunctionTok{model}\NormalTok{(}
    \AttributeTok{ETS =} \FunctionTok{ETS}\NormalTok{(KWH),}
    \StringTok{\textasciigrave{}}\AttributeTok{Additive ETS}\StringTok{\textasciigrave{}} \OtherTok{=} \FunctionTok{ETS}\NormalTok{(KWH }\SpecialCharTok{\textasciitilde{}} \FunctionTok{error}\NormalTok{(}\StringTok{"A"}\NormalTok{) }\SpecialCharTok{+} \FunctionTok{trend}\NormalTok{(}\StringTok{"A"}\NormalTok{) }\SpecialCharTok{+} \FunctionTok{season}\NormalTok{(}\StringTok{"A"}\NormalTok{)),}
    \AttributeTok{SNAIVE =} \FunctionTok{SNAIVE}\NormalTok{(KWH)}
\NormalTok{  )}

\CommentTok{\#forecast of 2013}
\NormalTok{power\_forecast }\OtherTok{\textless{}{-}}\NormalTok{ power\_fit }\SpecialCharTok{\%\textgreater{}\%} 
  \FunctionTok{forecast}\NormalTok{(}\AttributeTok{h =} \StringTok{"1 year"}\NormalTok{)}

\CommentTok{\#plot}
\NormalTok{power\_forecast }\SpecialCharTok{\%\textgreater{}\%}
  \FunctionTok{autoplot}\NormalTok{(power\_df2, }\AttributeTok{level =} \ConstantTok{NULL}\NormalTok{)}\SpecialCharTok{+}
  \FunctionTok{facet\_wrap}\NormalTok{( }\SpecialCharTok{\textasciitilde{}}\NormalTok{ .model, }\AttributeTok{scales =} \StringTok{"free\_y"}\NormalTok{) }\SpecialCharTok{+}
  \FunctionTok{guides}\NormalTok{(}\AttributeTok{colour =} \FunctionTok{guide\_legend}\NormalTok{(}\AttributeTok{title =} \StringTok{"Forecast"}\NormalTok{))}\SpecialCharTok{+}
  \FunctionTok{labs}\NormalTok{(}\AttributeTok{title=} \StringTok{"KWH Forecasts | Jan \textquotesingle{}13 {-} Dec \textquotesingle{}13"}\NormalTok{)}\SpecialCharTok{+}
  \FunctionTok{xlab}\NormalTok{(}\StringTok{"Month"}\NormalTok{) }\SpecialCharTok{+}
  \FunctionTok{ylab}\NormalTok{(}\StringTok{"KWH in Thousands"}\NormalTok{) }
\end{Highlighting}
\end{Shaded}

\includegraphics{DATA624_Project1_files/figure-latex/unnamed-chunk-60-1.pdf}

\begin{Shaded}
\begin{Highlighting}[]
\CommentTok{\#find ARIMA RMSE, MAE}
\FunctionTok{accuracy}\NormalTok{(power\_forecast\_diff, diff\_power) }\SpecialCharTok{\%\textgreater{}\%}
  \FunctionTok{select}\NormalTok{(.model, RMSE}\SpecialCharTok{:}\NormalTok{MAE)}
\end{Highlighting}
\end{Shaded}

\begin{verbatim}
## # A tibble: 2 x 3
##   .model      RMSE   MAE
##   <chr>      <dbl> <dbl>
## 1 ARIMA      1168.  774.
## 2 Auto ARIMA 1154.  771.
\end{verbatim}

\begin{Shaded}
\begin{Highlighting}[]
\CommentTok{\#find other RMSE, MAE}
\FunctionTok{accuracy}\NormalTok{(power\_forecast, power\_df2) }\SpecialCharTok{\%\textgreater{}\%}
  \FunctionTok{select}\NormalTok{(.model, RMSE}\SpecialCharTok{:}\NormalTok{MAE)}
\end{Highlighting}
\end{Shaded}

\begin{verbatim}
## # A tibble: 3 x 3
##   .model        RMSE   MAE
##   <chr>        <dbl> <dbl>
## 1 Additive ETS 1020.  626.
## 2 ETS          1050.  664.
## 3 SNAIVE       1036.  619.
\end{verbatim}

Additive ETS is the best model based on RMSE and MAE

\begin{Shaded}
\begin{Highlighting}[]
\CommentTok{\#reproduce the mode using the original dataset }
\NormalTok{power\_ETS\_fit }\OtherTok{\textless{}{-}}\NormalTok{ power\_df2 }\SpecialCharTok{\%\textgreater{}\%} 
  \FunctionTok{model}\NormalTok{(}\StringTok{\textasciigrave{}}\AttributeTok{Additive ETS}\StringTok{\textasciigrave{}} \OtherTok{=} \FunctionTok{ETS}\NormalTok{(KWH }\SpecialCharTok{\textasciitilde{}} \FunctionTok{error}\NormalTok{(}\StringTok{"A"}\NormalTok{) }\SpecialCharTok{+} \FunctionTok{trend}\NormalTok{(}\StringTok{"A"}\NormalTok{) }\SpecialCharTok{+} \FunctionTok{season}\NormalTok{(}\StringTok{"A"}\NormalTok{)))}

\CommentTok{\#generate the values}
\NormalTok{power\_ETS\_forecast }\OtherTok{\textless{}{-}}\NormalTok{ power\_ETS\_fit }\SpecialCharTok{\%\textgreater{}\%} 
  \FunctionTok{forecast}\NormalTok{(}\AttributeTok{h=}\DecValTok{12}\NormalTok{)}

\CommentTok{\#plot}
\NormalTok{power\_ETS\_forecast }\SpecialCharTok{\%\textgreater{}\%} 
  \FunctionTok{autoplot}\NormalTok{(power\_df2) }\SpecialCharTok{+}
  \FunctionTok{labs}\NormalTok{(}\AttributeTok{title =} \StringTok{"Monthly Residential Power Usage (Additive ETS |2024)"}\NormalTok{,}
       \AttributeTok{y =} \StringTok{"KWH in Thousands"}\NormalTok{)}
\end{Highlighting}
\end{Shaded}

\includegraphics{DATA624_Project1_files/figure-latex/unnamed-chunk-62-1.pdf}

\begin{Shaded}
\begin{Highlighting}[]
\NormalTok{(power\_forecast\_results }\OtherTok{\textless{}{-}} 
  \FunctionTok{as.data.frame}\NormalTok{(power\_ETS\_forecast) }\SpecialCharTok{\%\textgreater{}\%}
    \FunctionTok{select}\NormalTok{(DATE, .mean) }\SpecialCharTok{\%\textgreater{}\%} 
      \FunctionTok{rename}\NormalTok{(}\StringTok{\textquotesingle{}KWH Forecast\textquotesingle{}} \OtherTok{=}\NormalTok{ .mean))}
\end{Highlighting}
\end{Shaded}

\begin{verbatim}
##        DATE KWH Forecast
## 1  2014 Jan     9039.733
## 2  2014 Feb     8098.821
## 3  2014 Mar     7089.265
## 4  2014 Apr     6405.892
## 5  2014 May     6155.882
## 6  2014 Jun     7622.841
## 7  2014 Jul     8871.033
## 8  2014 Aug     9395.556
## 9  2014 Sep     8757.208
## 10 2014 Oct     6798.997
## 11 2014 Nov     6048.011
## 12 2014 Dec     7325.489
\end{verbatim}

\end{document}
